%! TEX root = ../main.tex

\section{Лекция от 23.03.2023}

ERP система --- это комплекс интегрированных программных инструментов для
создания единой базы данных, благодаря которым обеспечивается автоматизация
процессов планирования, учета, контроля и анализа всех ключевых аспектов
деятельности организации.

Этапы жизненного цикла программного обесечения ERP:
\begin{itemize}
  \item Каскадная модель;
  \item Поэтапная модель с промежуточным контролем;
  \item Спиральная модель.
\end{itemize}

\textbf{ПО}
\begin{itemize}
  \item алгоритм;
  \item структурирование данных.
\end{itemize}

\textbf{Каскадная модель} предусматривает последовательное выполнение всех
этапов проекта в строго фиксированном порядке. Переход на следующий этап
означает полное завершение работ на предыдущем этапе.\par

\textbf{Спиральная модель} предусматривает, что на каждом витке спирали
выполняется создание очередной версии продукта, уточнаяются требования проекта,
определяется его качество и планируются работы следующего витка.

\subsection{Задачи}
\begin{enumerate}
  \item Завод изготавливает светодиодные лампы. Известно, что для партии из 1000
    ламп потребляемое напряжение каждой распределено по нормальному закону
    характеристиками $M = 12.1\ В$, $СКО = 0.3\ В$. Расчитать вероятность того,
    что взятая наугад лампа будет потреблять напряжение $11.9\ В$.
    Формула распределения Гаусса:
    \begin{equation}
      P(x) = \frac{1}{\sigma \sqrt{2 \pi}} e^{- \frac{(x-M)^2}{2 \sigma^2}}
    \end{equation}
    Воспользуемся ей:
    \[
      P(x) = \frac{1}{0.3 \sqrt{2 \pi}} e^{- \frac{(11.9 - 12.1)^2}{2 (0.3)^2}}
    \] 
    \textbf{Это неправильно}\par
    Интеграл Лапласа:
    \begin{equation}
      \Phi(X) = \frac{1}{2\pi} \int\limits _0 ^X e^{-\frac{x^2}{2} dx}
    \end{equation} 
    \[
      X = \frac{x - M}{\sigma}
    \] 
    \[
      \Phi(X) = \Phi(0.66)
    \] 
    \[
      1 - \Phi(0.66) = 0.75
    \] 

    Математическое ожидание:
    \begin{equation}
      M = \sum_{i = 1} ^{n} x_i p_i
    \end{equation}
    Среднеквадратичное отклонение:
    \begin{equation}
      \sigma ^2 = \frac{1}{n - 1} \sum _{i=1} ^{n} (x_i - M)^2 p_i
    \end{equation} 

  \item Самолет имеет 4 турбовинтовых двигателя (2 на каждое крыло). Вероятность
    выхода из строя каждого из них равна $10^{-3}$. Известно, что самолет
    способен продолжать полет в случае выхода из строя трех двигателей. Найти
    вероятность крушения самолета.
    
    Ответ: $10^{-12}$

\end{enumerate}

\subsection{Производственный заказ}
\textbf{Производственный заказ} --- планово-учетная единица, объединяющая весь
комплекс работ по выполнению заказа.

Производственный заказ как планово-учетная единица включает весь комплекс работ,
от которых зависит достижение конечного результата --- выполнение заказа. Сюда
входит продукция конструкторских бюро, технологического отдела, производственных
цехов, работы по испытаниям в доводке.

Заказы на товары длительно пользования является показателем объема заказов,
которые имеются у производителей на товары длительного пользования. Под ними
подразумеваются те виды товаров, срок пользования которых от трех лет и выше.

Около $3 / 5$  всех заказов приходится на пассажирские и грузовые атомобили; $2
/ 5$  --- стройматериалы, мебель, бытовые товары.

Заказы на товары кратковременного пользования включают продукты питания, одежду,
товары легкой промышленности и т.д.

\paragraph{Технологические карты --- последовательность действий}
\begin{itemize}
  \item ТК позволяют спланировать производство материалов (изделий).
  \item ТК используются в качестве шаблона для производственных заказов и
    графика из выполнения; как основа для расчета стоимости.
  \item Ряд последовательных действий, необходимых для создания данного изделия.
  \item ТК отвечает на вопросы: Что? Где? Когда? Каким образом? 
\end{itemize}

\paragraph{Сетевой график}
\textbf{Сетевой график} --- это динамическая модель производственного процесса,
отражающая технологическую зависимость и последовательнось выполнения комплекса
работ, связывающая их свершение во времени с учетом затрат ресурсов и стоимости
работ с выделение при этом узких (критических) мест.
