%! TEX root = ../main.tex

\section{Лекция от 09.03.2023}
\subsection{Концепция управления проектированием и разработкой (Research \&
  Development)}

\textbf{НИОКР} --- полный цикл проектирования. Комплекс мероприятий, включающий
в себя как научные исследования так и производство опытных и мелоксерийных
образцов продукции, предшествующий запуску нового продукта в прошленное
производство.

\textbf{Научно-исследовательские работы}(НИР) --- работы поискового,
теоретического и экспериментального характера, выполняемые с целью определения
технической возможности создания новой техники в определенные сроки.

\begin{figure}[h]
  \centering
  \begin{tikzpicture}
    \node[block] (nir) {НИР};
    \node[block, below left of = nir, yshift=-1cm, xshift=-1cm] (pr) 
      {Прикладное};
    \node[block, below right of = nir, yshift=-1cm, xshift=1cm] (teor) 
      {Теоретичское};
    \draw[arrow] (nir) -- (pr);
    \draw[arrow] (nir) -- (teor);
  \end{tikzpicture}
\end{figure}

\textbf{Опытно-конструкторские работы} (ОКР) --- комплекс работ по разработке
конструкторской и технологической документации на опытный образец.

\paragraph{Жизненный цикл иделия}\mbox{}\par
\begin{enumerate}
  \item НИР
  \item Техническое предложение
  \item ТЗ на ОКР
  \item ОКР
  \item Разработка и изготовление опытных образцов
  \item Испытание и контроль
  \item Технологическая подготовка
  \item Произовдство
  \item Эксплуатация, ремонт
  \item Утилизация
\end{enumerate}

\paragraph{Этапы оптыно-конструкторских работ}
\begin{center}
  \begin{tikzpicture}[node distance = 1.5cm]
    \node[process] (p1) {ТЗ на ОКР}; 
    \node[process, below of = p1] (p2) {ТЗ на составные части ОКР};
    \node[process, below of = p2] (p3) {Эскизный проект};
    \node[process, below of = p3] (p4) {Технический проект};
    \node[process, below of = p4] (p5) {Разработка рабочей документации};
    \node[process, below of = p5, yshift=-0.5cm, fill=yellow!90!black] (p6) 
      {Изготовление и испытания опытных образцов};
    \node[process, right of = p3, xshift = 4cm, fill=green!90!black] (l1)
      {Проектирование};
    \node[process, right of = p5, xshift = 4cm, yshift=1cm, fill=red!90!black] (l2) 
      {Конструирование};
    \draw[arrow] (p1) -- (p2);
    \draw[arrow] (p2) -- (p3);
    \draw[arrow] (p3) -- (p4);
    \draw[arrow] (p4) -- (p5);
    \draw[arrow] (p5) -- (p6);
    \draw[arrow] (l1.180) -- (p2.0);
    \draw[arrow] (l1) -- (p3);
    \draw[arrow] (l1.180) -- (p4.0);
    \draw[arrow] (l2.180) -- (p4.0);
    \draw[arrow] (l2.180) -- (p5.0);
  \end{tikzpicture}
\end{center}
  
\paragraph{Рабочая документация}\mbox{}\par
Изначально производится разработка рабочей документации на изготовление и
испытание опытного образца, формирование комплекта конструкторских документов в
следующей послдеовательности.

\paragraph{Схемы}
\begin{itemize}
  \item По типу
    \begin{enumerate}
      \item Общие
      \item Функциональные
      \item Принципиальные
      \item Соединений
      \item Подключений
    \end{enumerate}
  \item По характеру
    \begin{enumerate}
      \item Электрические
      \item Кинематические
      \item Гидравлические
      \item Пневматические
    \end{enumerate}
\end{itemize}

\paragraph{Испытание и доводка}
Предварительные испытания производятся с целью проеверки соответствия опытного
образца требованиям ТЗ и определения возможности его появления на окончательные
(государственные, ведомственные или внутрикорпоративные испытания).
Предварительные испытания включают в себя:
\begin{itemize}
  \item предъявительские испытания;
  \item приемо-сдаточные испытания;
  \item предварительные испытания.
\end{itemize}

\paragraph{Эволюция стандартов планирования}\mbox{}\par
\begin{enumerate}
  \item MRP (Material Resource Planning)
  \item MRPII (Manufacturing Resource Planning)
  \item ERP (Enterprise Resource Planning)
  \item CSRP (Customer Synchronized Resource Planning)
  \item S. PLM
\end{enumerate}

MRP --- методология планирования потребностей в материалах. Суть в том, чтобы
минимизировать издержки, связанные со складскими запасами и на различных
участках в производстве.\par

Задачи MRP
\begin{enumerate}
  \item Минимизация запасов сырья.
  \item Оптимизировать поступление материалов и комплектующих в производство
  \item Исключить простои оборудования
\end{enumerate}

Недостатки MRP
\begin{enumerate}
  \item Отсутствие контрля выполнения плана закупок и механизма корректировки
    этого плана в случае возникновения ситуаций, мешающих его нормальному
    исполнению.
  \item Ограниченный учет производственных факторов.
\end{enumerate}
