%! TEX root = ../main.tex

\section{Лекция от 16.03.2023}

\subsection{MRP II}

\textbf{MRP II} (manufacturing resource planning~--- планирование
производственных ресурсов)~--- стратегия производственного планирования,
обеспечивающая как операционное, так и финансовое планировнаие производства,
обеспечивающая более широкий охват ресурсов предприятия, нежели MRP.

MRP II --- методология, направленная на эффективное управление всеми ресурсами
производственного предприятия (сырья, материалов, оборудования, персонала и
т.д.). 

Планирование производственных ресурсов:
\begin{itemize}
  \item Цель: Планирование и мониторинг всех ресурсов производственной компании
    (замкнутый цикл):
    \begin{itemize}
      \item Производство
      \item Маркетнг
      \item Финансы 
      \item Инженеринг
    \end{itemize}
\end{itemize}

\paragraph{Замкнутый цикл MRP}
\begin{center}
  \begin{tikzpicture}[node distance = 2cm]
    \node[rectangle, draw = black, text width = 8cm, fill=black!10!yellow] (top)
      {Производственное планирование\\ График производства\\ Требования по
      планированию материалов\\ Требования по загрузке производства};
    \node[diamond, draw = black, below of = top, yshift=-2cm,
      fill=yellow!30!green] (mid) {Реалистичность?};
    \node[rectangle, draw = black, text width = 8cm, below of = mid, yshift =
      -2cm, fill=black!10!yellow] (bot) {Исполнение:\\ Планы по загрузке
      производства\\ Планы по снабжению};
    \draw[arrow] (top) -- (mid);
    \draw[arrow] (mid) -- node[anchor=west]{Да}(bot);
    \draw[arrow] (mid.180) -| node[anchor=north]{Нет}(top.200);
    \draw[arrow] (bot.0) -| node[anchor=west,yshift=2cm, text width = 2cm]
      {Обратная связь} ([xshift=1cm]top.east) --  (top.east);
  \end{tikzpicture}
\end{center}

\newpage

\paragraph{Формула Байеса}\mbox{}\par
\[
  P(\theta | Data) = \frac{P(Data | \theta) \cdot P(\theta)}{P(Data)}
\] 

\textbf{Задача:}\par
На фабрике производят монтаж светодиодов печатной лампы. Первый монтажник делает
эту операцию с вероятностью брака $0.1$, второй с $0.2$. Первый монтажник
совершает 30 подобных операций в час. Второй 20. Определить вероятность того,
что бракованное изделие сделано вторым монтажником.
\[
  P(М_1) = \frac{30}{20+30} = 0.6
\] 
\[
  P(М_2) = \frac{20}{20+30} = 0.4
\] 
\[
  P(Б | М_1) = 0.1
\] 
\[
  P(Б | М_2) = 0.2
\] 
\[
  P(Б) = P(Б | М_1) \cdot P(М_1) + P(Б | М_2) \cdot P(М_2) = 0.06 + 0.08 = 0.14
\] 
Надо найти $P(М_2 | Б)$
\[
  P(М_2 | Б) = \frac{P(Б | М_2) \cdot P(М_2)}{P(Б)} = \frac{0.08}{0.14} = 0.57
\] 

\subsection{ERP}
\textbf{ERP}~--- (Enterprise Resource Planning) планирование ресурсов
предприятия. Организацияонная стратегия интеграции производства и операций,
управления трудовыми ресурсами, финансового менеджмента и управления активами,
ориентированная на непрерывную балансировку и оптимизацию ресурсов предприятия
посредством спецализорованного интегрированного пакета прикладного программного
обеспечения, обеспечивающего общую модель данных и процессов для всех сфер
деятельносьти.

\textbf{Характеристики ERP-системы:}
\begin{itemize}
  \item интеграция основных производственных процессов организации;
  \item обработка большинства бизнес-операций;
  \item единовременное сохранение каждой единицы информации в общей базе данных
    предприятия для последующего ее использования;
  \item обеспечение доступа к базе данных в режиме реального времени;
  \item интеграция обработки деловых операций и действий по планированию в
    случае необходимости;
  \item применение как в традиционной внутренней среде <<Клиент --- сервер>>,
    так и во внешней среде с привлечением Интрнета-технологий
  \item поддержка различных отраслей народного хозяйства;
  \item возможность настройки системы с учетом специфических нужд предприятия
    при остутствии навыков программирования;
  \item поддержка нескольких языков и иностранных валют.
\end{itemize}

\textbf{Модули ERP}
\begin{table}[h]
  \centering
  \begin{tabular}{|c|c|c|}
    \hline
     \textbf{Финансы} & \textbf{Персонал} & \textbf{Операции} \\
    \hline
     Бухгалтерские & Кадровый учет & Логистические \\
     учетно-управленческие & оценка персонала  & Производственные  \\
     финансово-управленческие & подбор персонала  & Обеспечивающие \\
     & & Сбытовые \\
     \hline
  \end{tabular}
\end{table}

\textbf{Задача:}\par
Станок-автомат штампует детали. Вероятность того, что изготовленная деталь
окажется бракованной равна 0.001. Найти вероятность того, что среди 350 деталей
окажется ровно 3 бракованных.
