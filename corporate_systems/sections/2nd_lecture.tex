%! TEX root = ../main.tex

\section{Лекция от 16.02.2023}
\subsection{Система менеджмента качества}
СМК --- совокупность организационной структуры, методик, процессов и русурсов,
необходимых для общего руководства качеством. Она предназначена для постоянного
улучшения деятельности, для повышения конкурентоспособности организации на
национальном и мировом рынках, определяет конкурентоспособность любой
организации.

СМК основана на восьми принципах менеджмента качества:
\begin{enumerate}
  \item Ориентация на потребителя
  \item Лидерство руководителя
  \item Вовлечение персонала 
  \item Процессный подход
  \item Системный подход к менеджменту
  \item Постоянное улучшение
  \item Принятие решения, основанных на фактах
  \item Взаимовыгодные отношения с поставщиками
\end{enumerate}

\paragraph{Цикл Деминга (PDCA)}\mbox{}\par
\textit{Plan $\to$ do $\to$ Check $\to$ Act}
\begin{enumerate}
  \item Планирование --- установление целей и процессо, необходимых для
    достижения целей;
  \item Выполнение запланированных работ.
  \item Проверка сбор информации и контроль результата на основе ключевых
    показателей эффективности;
  \item Воздействие (управление, корректировка) принятие мер по устранению
    причин отклонений от запланированного результата.
\end{enumerate}

ISO 9000 --- серия международных стандартов, описывающих требования к системе
менеджмента качества организаций и предприятий:
\begin{enumerate}
  \item ISO 9000 Словарь терминов о системе менеджмента качества;
  \item ISO 9001 Содержит набор требований к системам менеджмента качества;
  \item ISO 9004 Содержит руководство по достижению устойчивого успеха любой
    организацией в сложной и постоянно изменяющейся среде;
  \item ISO 19011 Стандарт, описывающий методы проведения аудита в системах
    менеджмента, в том числе, менеджмента качества
\end{enumerate}

\textbf{Продукция }--- материальный объект, который является результатом
деятельности не требующей для своего осуществления прямого взаимодействия между
поставщиком и потребителем.

\textbf{Услуга} --- нематериальный объект, который является результатом по
меньшей мере, одного действия, требующейго для своего осуществляения прямого
взаимодействия между поставщиком и потребителем.

\textbf{Структура СМК:}
\begin{enumerate}
  \item Модель бизнес-процессов верхего уровня, в которые интегрированы
    требования СМК
  \item Уровень департаментов
  \item Уровень управлений
  \item Уровень отделов
  \item Уровень автоматизации деятельности отделов
\end{enumerate}

\textbf{Риск} --- влияние неопределенности на ожидаемый результат.

\paragraph{Поддерживающая деятельность.}\mbox{}\par

Организация должна определить знания, необходимые для функционирования ее
процессов и для достижения соответствия продукции и услуг.

Эти знания должны поддерживаться на соответствующем уровне и быть доступными для
распространения в необходимом объеме.

При рассмотрении вопросов, касающихся изменений потребностий или тенденций,
организация должна рассмотреть свои имеющиеся знания и определить, каким образом
она будет приобретать необходимые дополнительные знания или получит к ним
доступ.

\paragraph{Оценочная деятельность}\mbox{}\par

Общие положения:\\
Организация должна определить:
\begin{enumerate}
  \item что необходимо подвергнуть мониторингу и измерениеям;
  \item прменимые методы мониторинга, измерений, анализа и оценивания,
    обеспечивающая признание из результатов;
  \item когда должны проводиться мониторинг и измерения;
  \item когда результаты мониторинга и измерений должны быть проанализированы и
    оценены.
\end{enumerate}

\subsection{Концепция управления производством}
\begin{figure}[h!]
  \centering
  \begin{tikzpicture}[node distance=3cm]
    %\node (app) [process, below of = BIOS] {Аппаратура};
    \node[process] (prom) {Промышленное предприятие};
    \node[process, below left of = prom, xshift = -8] (proizv) {Производство};
    \node[process, below right of = prom, xshift = 8] (project) {Проектирование и разработка};
    \draw[arrow] (prom) -- (proizv);
    \draw[arrow] (prom) -- (project);
  \end{tikzpicture}
  \label{fig:}
\end{figure}

\paragraph{Типы производства}
\textbf{Типы производства} --- это категорийность производства продукции по
видам организации структуры производственных факторов в отношении количества
самого продукта или услуги. В машиностроении определяется в зависимости от
коэффициента закрепления операций.

Тип производства опредеяется согласно ГОСТ 3.1108-74
\[
  K = \frac{N}{P_m}
\] 

Где $N$ --- число различных операций, выполняемых в течение календарного
времени;\\
$P_m$  --- число рабочих мест, на которых выполняются данные операции.\\
Таким образом коэффициент закрепления операций:
\begin{enumerate}
  \item Единичное прозводство --- больше $40$
  \item Мелкосерийное производство --- $20\dots40$
  \item Среднесирийное --- $10\dots20$ 
  \item Крупносерийное --- $1\dots10$ 
  \item Массовое --- $1$
\end{enumerate}

\textbf{Автоматизация производства} --- процесс в развитии машинного
производства, при котором функции управления и контроля, ранее выполнявшиеся
человеком, передаются приборам и автомтическим устройствам.

\textbf{Спецификация} --- основной конструкторский документ, определяющий состав
сборочной единицы, комплекса, комплекта.

\begin{figure}[htpb]
  \centering
  \begin{tikzpicture}[auto, node distance=3cm]
    \node[process] (itr) {ИТР};  
    \node[process, below right of = itr, xshift=8] (dev) {Инженер-разработчик};
    \node[process, below left of = itr, xshift =-8] (constr) {Инженер-конструктор};
    \node[process, below of = itr, yshift=-18] (tech) {Инженер-технолог};
    \node[process, below right of = dev, xshift=8] (prog) {Инженер-программист};
    \draw[arrow] (itr) -- (dev);
    \draw[arrow] (itr) -- (constr);
    \draw[arrow] (itr) -- (tech);
    \draw[arrow] (itr) -| (prog);
  \end{tikzpicture}
\end{figure}

\textbf{Производственный процесс} --- это целенаправленное, поэтапное
превращение исходного сырья и материалов в готовый продукт заданного свойства и
пригодный к потреблению или к дальнейшей обработке.

\textbf{Технологический процесс} --- это упорядоченная последовательность
взаимосвязанных действий, выполняющих целенаправленные действия по изменению и
(или) определению состояния предмета труда.

\begin{figure}[htpb]
  \centering
  \begin{tikzpicture}[node distance = 2cm]
    \node[process] (proizv) {Производственный\\ процесс};
    \node[process, right of = proizv, xshift=100] (tech) {Технологический\\
      процесс};
    \node[process, below of = proizv] (izd) {получение изделия};
    \draw[double, arrow] (tech) -- (proizv);
    \draw[arrow] (proizv) -- (izd);
  \end{tikzpicture}
\end{figure}

\paragraph{Документирование технологического процесса}
\begin{itemize} 
  \item \textbf{Маршрутная карта} --- описание маршрутов движения
    по цеху изготовляемой детали.
  \item \textbf{Операционная карта} --- перечень переходв, установок и применяемых
    инструментов.
  \item \textbf{Технологическая карта} --- документ, в котором описан: процесс
    обработки деталей, материалов, конструкторская документация, технологическая
    оснастка.
\end{itemize}

\paragraph{Методология автоматизации производства}
\begin{enumerate}
  \item Оганизация и планирование проекта.
  \item Диагностика существующей системы управления предприятием.
  \item Совершенствование бизнес-процессов и функций предприятий.
  \item Внедрение и управление корпоративным документооборотом и деловыми
    процессами.
  \item Проектирование КИС.
  \item Построение макета КИС.
  \item Проведение пилотной эксплуатации КИС.
  \item Промышленная эксплуатация.
  \item Анализ эффективности внедрения КИС.
\end{enumerate}
