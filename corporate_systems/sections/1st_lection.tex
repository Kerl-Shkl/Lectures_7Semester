%! TEX root = ../main.tex

\section{Лекция от 09.02.2023}
\paragraph{Технические революции:}\mbox{}\par
\begin{enumerate}
  \item Неолитическая (стали применять изделия из металла);
  \item Промышленная (фабричный, монофактурный метод производства XV--XVI\,вв);
  \item Научно-техническая (повсеместное использования исследований).
\end{enumerate}

\paragraph{Факторы организационной структуры управления:}\mbox{}\par
\begin{itemize}
  \item масштабы бизнеса (малый, средний, большой);
  \item производственные и отрослевые особенности предприятия;
  \item характер производства (массовый, серийный, единичный);
  \item сфера деятельности фирм (местный, национальный, внешний рынок);
  \item уровень механизации и автоматизации управленческих работ;
  \item квалификация работников.
\end{itemize}

\paragraph{Типы структур управления:}\mbox{}\par
Структуры управления на многих современных предприятиях были построены в
соответствии с принципами управления, сформированными еще в начале XX века:
\begin{enumerate}
  \item Принцип иерархичности уровней управления, при котором каждый
    нижестоящий уровень контролируется вышестоящим и подчиняется ему;
  \item Принцип разделения труда на отдельные функции и специализации
    работников по выполянмым функциям;
  \item Вытекающий из него принцип обезличенности выполнения работниками
    своих функций;
  \item Принцип квалификационного отбора.
\end{enumerate}

\paragraph{Линейный тип организации}\mbox{}\par
Преимущества линейной структуры:
\begin{itemize}
  \item четкая система взаимных связей функций и подразделений;
  \item четкая система единоначалия --- один руководитель сосредотачивает в
    своих руках руководство всей совокупностью процессов, имеющих общую цель;
  \item ясно выраженная ответственность;
  \item быстрая реакция исполнительных подразделений на прямые указания
    вышестоящих.
\end{itemize}

Недостатки линейной структуры:
\begin{itemize}
  \item отсутствие звеньев, занимающихся вопросами стратегического планирования;
  \item тенденция к волоките и перекладыванию ответственности при решении
    проблем, требующих учатия нескольких подразделений;
  \item малая гибкость и приспособляемость к изменению ситуации;
  \item критерии эффективности и качества работы подразделений и организации в
    целом --- разные.
\end{itemize}

\paragraph{Матричный тип организации}\mbox{}\par
Характерно, что существуют столбцы в которых выделяют конкретные задачи, а в
строках выделяют инструменты необходимые для решения этих задач. Затем на эту
задачу назначают сотрудника. %TODO insert photo
Преимущества:
\begin{enumerate}
  \item Лучшая ориентация на проектные (или программные) цели и спрос;
  \item более эффективное текущее управление;
  \item возможность снижения расходов и повышения эффективности использования
    ресурсов;
  \item улучшения контроля за отдельными задачами проекта или целевой программы.
\end{enumerate}

Недостатки:
\begin{enumerate}
  \item Трудность установления четкой ответственности за работу по заданию
    подразделения и по заданию проекта или программы;
  \item высокие требования к квалификации, личным и деловым качествам
    работников;
  \item возможность нарушения правил и стандартов, принятых в функциональных
    подразделениях.
\end{enumerate}

\subsection{Корпоративные информационные системы}
Корпоративная информационная система --- это открытая интегрированная
автоматизированная система реального времени по автоматизации бизнес-процессов
компании всех уровней, в том чиисле, и бизнес-процессов принятия управленческих
решений. При этом степень автоматизации бизнес-процессов определяется исходя из
обеспечения масимальной прибыли компании. ($P \to max$, $P$ --- profit (прибыль))

\paragraph{Признаки корпоративных информационных систем}\mbox{}\par
\begin{enumerate}
  \item Соответствие потребностям компании, бизнесу компании, согласованность с
    организационно-финаносовой структурой компании.\\
    Наличие документов, регламентирующих работу по финансовым, юридическим и
    техническим условиям законодательству РФ. Начилие структурных связей между
    подразделениями.
  \item Интегрированность.\\
    Сквозная автоматизированная система, в которой каждому отдельному модулю
    системы в реальном времени доступна вся необходимя информация, вырабатываемя
    другими модулями.
  \item Открытость и масштабируемость.\\
    Корпоративная информационная система должна быть открытой для включения
    дополнительных модулей и расширения системы как по масштабам и функциям.
\end{enumerate}

Преимущества внедрения КИС:
\begin{enumerate}
  \item Получение достоверной оперативной информации о деятельности всех
    подразделений компании;
  \item повышение эффективности управления компанией;
  \item сокращение хатрат рабочего времени на выполнение рабочих операций;
  \item повышение общей результативности работы за счет более рациональной ее
    организации.
\end{enumerate}

\paragraph{Классификация КИС:}\mbox{}\par
\begin{enumerate}
  \item Финансово-управленические системы. Предназначены для ведения учета по
    одному или нескольким направлениям (бухгалтерия, сбыт, склад, кадры и т.д.)
  \item Производственные системы (также называемые системами производственного
    управления). Они предназначены в первую очередь для управления и
    планирования производственного процесса: серийное сборочное (электроника,
    машиностроение), мелкосерийное и опытное (авация, тяжелое машиностроение),
    дискретное (металлургия, химия, упаковка), непрерывное (нефтедобыча).
  \item Научно-исследовательские системы (управление НИОКР). Определяют процессы
    разработки новых продуктов на рынке. Структурно подразделяются на НИР
    (исследования, расчеты и заключения) и ОКР (макетирование, проектирование,
    настройка).
\end{enumerate}

\paragraph{Ресурсы корпораций}\mbox{}\par
\begin{itemize}
  \item материальные (материалы, готовая продукция, основные средства)
  \item финансовые (денежные ресурсы, кредита)
  \item людские (персонал и его квалификация)
  \item знания (исследования)
\end{itemize}

\paragraph{Этапы проектирования КИС}\mbox{}\par
\begin{itemize}
  \item Анализ
  \item Проектирование
  \item Разработка
  \item Интеграция и тестирование
  \item Внедрение
  \item Сопровождение
\end{itemize}
