%! TEX root = ../main.tex

\section{Лекция от 06.04.2023}
\textbf{Сетевое планирование} --- метод анализа сроков в управлении проектами,
позволяющие спрогнозировать временные сроки исполнения и оценить
финансово-экономические издержки.

Основным плановым документом в системе сетевого планирования и управления (СПУ)
является сетевой график (сетевая модель), представляющий собой
информационно-динамическую модель, в которой изображаются взаимосвязи и
результаты всех работ необходимы для достижения конечной цели разработки.

\subsection{Методика сетевого планирования были разработаны в конце 50-х годов в
  США}
В 1956 году в военно-морских силах США был создан метод анализа и оценки
программ PERT. Данный метод был разработан корпорацией "Локхилд".\par
\textbf{Метод PERT} --- метод оценки и проверки программ Исходя из положений
метода PERT длительность каждой операции имеет пределы, которые исходят из
статистического распределения.

Входом для данного метода оценки служит список элементарных пакетов работ.
Диапазон неопределенности достаточно охарактеризовать тремя оценками:
\begin{itemize}
  \item $M_i$  --- наиболее вероятная оценка трудозатрат;
  \item $O_i$  --- минимально возможные трудозатраты на реализацию пакета работ.
    Ни один риск не реализовался. Вероятность равна 0;
  \item $P_i$  --- пессимистическая оценка трудозатрат. Все риски реализовались.
\end{itemize}
Оценку средней трудоемкости по каждому элементарному пакету можно определить по
формуле:
\[
  E_i = (P_i + 4M_i + O_i) / 6
\] 

Для расчета среднеквадратичного отклонения используется формула:
\[
  СКО_i = \frac{P_i - O_i}{6}
\] 

Суммарная трудоемкость проекта может быть рассчитана по формуле:
\[
  E = \sum E_i
\] 

СКО для оценки суммарной трудоемкости будет составлять:
\[
  СКО = \sqrt{\sum СКО_i^2} 
\] 

Тогда для оценки суммарной трудоемкости проекта, которую мы не превысим с
вероятностью $95\%$, можно применить формулу:
\[
  E_{95\%} = E + 2 \cdot СКО
\] 

\subsection{Этапы процесса обработки данных}
\begin{enumerate}
  \item Подготовка данных
  \item Анализ данных
  \item Представление знаний 
\end{enumerate}

\textbf{Анализ данных} --- область математики и информатики, занимающаяся
построением и исследованием наиболее общих математических методов и
вычислительных алгоритмов извлечения знаний из экспериментальных (в широком
смысле) данных; процесс исследования, фильтрация, преобразования и моделирования
данных с целью извлечения полезной информации и принятия решения.

\textbf{Интеллектуальный анализ данных} --- это особый метод анализа данных,
который фокусируется на моделировании и открытии данных, а не на их описании.

\textbf{OLAP}(online analytical processing) --- технология обработки данных,
заключающаяся в подготовке суммарной (агрегированной) информации на основе
больших массивов данных, структурированных по многомерному принципу.

Агрегатные функции образуют многомерный (и, следовательно, не реляционный) набор
данных (называемый гиперкубом или метакубом), оси которого содержат параметры, а
ячейки --- зависящие от них агрегатные данные. Вдоль каждой оси данные могут
быть организованы в виде иерархии, представляющей различные уровни их
детализации.

\paragraph{Обработка тестовых данных}\mbox{}\par
\begin{itemize}
  \item поиск (в т.ч. по ключевым словам)
  \item семантический анализ
  \item тематическая и жанровая классификация сообщений на основе
    лексико-статического анализа (в т.ч. фильтрация спама)
  \item отбор сообщений на основе структурно-статистических признаков
  \item оценка достоверности
  \item сокращения избыточности представления (реферирование и аннотирование)
  \item сжатие текстовых данных
\end{itemize}
