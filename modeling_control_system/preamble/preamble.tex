%! TEX root = ../main.tex

\documentclass[a4paper,12pt]{article}

%%% Работа с русским языком
\usepackage{cmap}					% поиск в PDF
\usepackage{mathtext} 				% русские буквы в формулах
\usepackage[T2A]{fontenc}			% кодировка
\usepackage[utf8]{inputenc}			% кодировка исходного текста
\usepackage[english,russian]{babel}	% локализация и переносы
\usepackage{indentfirst}            % красная строка в первом абзаце
\frenchspacing                      % равные пробелы между словами и предложениями

%%% Дополнительная работа с математикой
\usepackage{amsmath,amsfonts,amssymb,amsthm,mathtools} % пакеты AMS
\usepackage{icomma}                                    % "Умная" запятая

%%% Свои символы и команды
\usepackage{centernot} % центрированное зачеркивание символа
\usepackage{stmaryrd}  % некоторые спецсимволы


%%% Работа с картинками
\usepackage{graphicx}    % Для вставки рисунков
\setlength\fboxsep{3pt}  % Отступ рамки \fbox{} от рисунка
\setlength\fboxrule{1pt} % Толщина линий рамки \fbox{}
\usepackage{wrapfig}     % Обтекание рисунков текстом

%%% Работа с таблицами
\usepackage{array,tabularx,tabulary,booktabs} % Дополнительная работа с таблицами
\usepackage{longtable}                        % Длинные таблицы
\usepackage{multirow}                         % Слияние строк в таблице

%%% Оформление страницы
\usepackage{extsizes}     % Возможность сделать 14-й шрифт
\usepackage{geometry}     % Простой способ задавать поля
\usepackage{setspace}     % Интерлиньяж
\usepackage{enumitem}     % Настройка окружений itemize и enumerate
\setlist{leftmargin=25pt} % Отступы в itemize и enumerate

\geometry{top=25mm}    % Поля сверху страницы
\geometry{bottom=30mm} % Поля снизу страницы
\geometry{left=20mm}   % Поля слева страницы
\geometry{right=20mm}  % Поля справа страницы

\setlength\parindent{15pt}        % Устанавливает длину красной строки 15pt
\linespread{1.3}                  % Коэффициент межстрочного интервала
%\setlength{\parskip}{0.5em}      % Вертикальный интервал между абзацами
%\setcounter{secnumdepth}{0}      % Отключение нумерации разделов
%\setcounter{section}{-1}         % Нумерация секций с нуля
\usepackage{multicol}			  % Для текста в нескольких колонках
\usepackage{soulutf8}             % Модификаторы начертания

%%% Содержаниие
\usepackage{tocloft}
\tocloftpagestyle{main}
%\setlength{\cftsecnumwidth}{2.3em}
%\renewcommand{\cftsecdotsep}{1}
%\renewcommand{\cftsecpresnum}{\hfill}
%\renewcommand{\cftsecaftersnum}{\quad}

%%% Шаблонная информация для титульного листа
\newcommand{\CourseName}{Моделирование систем управления}
\newcommand{\FullCourseNameFirstPart}{\so{МОДЕЛИРОВАНИЕ СИСТЕМ УПРАВЛЕНИЯ}}
\newcommand{\SemesterNumber}{V}
\newcommand{\LecturerInitials}{Ерофеев Сергей Анатольевич}
\newcommand{\CourseDate}{осень 2022}
\newcommand{\AuthorInitials}{Шкалин Кирилл Павлович}

%%% Колонтитулы
\usepackage{titleps}
\newpagestyle{main}{
	\setheadrule{0.4pt}
	\sethead{\CourseName}{}{}
	\setfootrule{0.4pt}                       
	\setfoot{ИКНТ УТС, \CourseDate}{}{\thepage} 
}
\pagestyle{main}  

