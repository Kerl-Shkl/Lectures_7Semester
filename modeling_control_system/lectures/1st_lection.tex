%! TEX root = ../main.tex
\section{Лекция от 3.9.2022. Введение в моделирование.}
\subsection{Введение в программно ореиентированное моделирование. Знакомство с
терминами.}
Под \textbf{моделированием} понимается замещение объекта исследования на 
определенный образ, в необходимой мере передающий свойствово объекта т.е. модель 
и проведения эксперемента над этой моделью.\par
Набор данных (входные, выходные, функция связи).\\
{\ttfamilyразные модели:
  \begin{itemize}
    \item структурная схема (состав)
    \item функциональная схема (состав + параметры)
  \end{itemize}}
\textbf{Эксперемент} --- вычесления (на вход подаем данные, смотрим
результат).\par
Выходные данные такого вычислительного эксперемента считаются \textbf{расчетными
характеристиками объекта} исследования.\par
Чтобы переложить исследования на компьютер наобходимо какой-то мере раскрыть
изучаемый процесс, а именно определить множество параметров и обозначить их
взаимосвязь, где под параметрами понимаются величина, позволяющие адекватно (в
должной мере) передать механизм протекания процесса в реальном объекте.\par
Сформированная таким образом модель (раскрыли параметры) должна бать
формализована т.е. описана на формальном языке (логика + математика),
объективность синтаксиса которого делаем это описание безусловно понятным и
однозначно трактуемым. (Без хаоса)\par
Формально информационная модель, переведенная на язык математических переменных
и соотношений - \textbf{математическая модель}.
Не всем объектам и явлениям можно сопоставить формально информационные модели.
Напрмер: динамическая труба, завихрения, турбулентность.\par
На ряду с компьютерным моделированием выполняется физическое моделирование, в
рамках которого исследования выполяются не в виртуальном, а в материальном мире
на системе, которая эквивалентна изучаемому объекту с точностью до полного или
частичного воспроизведения физических свойств.\par
\begin{enumerate}
  \item Например работа системы энергоснабжениям города моделируют с пом
    специальной электрической схемы. Модель + тестовый комплекс.
  \item А информацию об аэродинамических свойствах летательного аппарата
    получают из результатов продуваемой модели а аэродинамической трубе.
\end{enumerate}\par
Иногда ставятся смешанные/полунатурные эксперименты, в которых части объекта
заменяются аппаратным аналогом (дешевым), к которому подсоединяются
вычислительные устройства, иммитирующие работу остальной части объекта на уровне
отработки ее математического образа.\par
Синхронизация двух составляющих такого рода смешанной модели выполяется с
помощью различного рода согласующих устройств (ЦАП/АЦП).\par
Смешанные эксперименты часто проводят для ПО (стиральные машины)\par
Для дальнейшего изложения целесообразно дать математической модели более строгое
определение, как набор охарактеризущих объект \textbf{переменных состояния} и
соотношений между ними. (При поиске уравнений учитываются промежуточные
переменные.)\par
В конечном итоге расчет математической модели сводится к разрешению уравнения
или системы уравнений аналитическим или численным методом. Методы: метод
Гаусса.\par
Изложение схемы расчетов выходных параметров модели по ее входным данным в виде
алгоритма с привязкой к конкретному методу решения называется
\textbf{алгоритмической моделью}.\par
Реализация алгоритмической модели на конкретном языке программирования
называется \textbf{алгоритмическим процессом}.\par
