%! TEX root = ../main.tex

\section{Лекция от 29.10.2022}
\subsection{S функция для непрерывного состояния}
Блок \verb!S-function!.
\[
  \left\{\begin{aligned} 
    &y = F_1(t, x, u)\\
    &x = x_{н} + x_y\\
    &\dot x_{н} = F_2 (t, x, u)\\
    &x_y ^{k+1} = F_3(t, x, u) \\
  \end{aligned}\right. 
\] 
Есть поле \verb!S-function name! и поле \verb!S-function parameters!\par

Надо найти файл sfuntmpl.m и его поменять:
\begin{verbatim}
  function[sys, x0, str, ts] = sfuntmpl(t, x, u, flag) %В квадратных, что на выходе
                                                       %В круглых, что на входе
  %Чтобы доавить параметры заголовок надо поменять
  function[sys, x0, str, ts] = s1(t, x, u, flag, p_1, p_2, p_n) 

  switch-case ...

  1. mdlInitializeSizes
  /*
  sizes.NumInputs - количество входов
  sizes.NumContStates - Количество непрерывных переменных состояний
  sizes.NumOutputs - количество выходов
  sizes.DirFeedThrough - под прямым проходом для непрерывных систем
                         подразумевается явное присутствие в уравнении 
                         y = F_1(t, x, u) входного сигнала или его компонентов.
                         Под отсутствием прямого прохода подразумевается или
                         отсутствие зависимости y от u или зависимость y от u
                         только через x. Если зависимости нет, ставим 0. Если
                         зависимость есть, задаем 1. Используется при проверке
                         наличия замкнутых алгебраических контуров.
  Также указываем размерности переменных входа и состояния. Если передать -1, то
  будет автоматическое определение этих величин по размерности поступающего
  входного сигнала. Dynamically sized inputs - динамические входные величины.
  */
  
  %метод оформлен в виде функции со следующим заголовком
  function[sys, x0, str, ts] = mdlInitializeSizes %ничего на вход не принимает
  /*
  sys - структура класса SimStruct 
  ts - определяет схемму квантования. На непрерывных моделях не нужен.
  */
  %код:
  sizes = simsizes;
  sizes.NumContStates = 0;
  sizes.NumContStates = 0;
  sizes.NumDiscStates = 0;
  sizes.NumOutputs = 0;
  sizes.Inputs = 0;
  sizes.DirFeedThrough = 1;
  sizes.NumSampleTimes = 1; %Количество строк в массиве ts

  sys = simsizes(sizes);
  x0 = [];
  str = [];
  ts = [0 0];

  %--------------------------------
  %Два метода для непрерывных моделей
  function sys = mdlDerivatives(t, x, u) %Соответствует уравнению \dot X = AX + BU
  
  function sys = mdlOutputs(t, x, u) %Соответствует Y = CX + DU

  switch flag,
    case 0,
      [sys,x0,str,ts,simStateCompliance]=mdlInitializeSizes;
    case 1,
      sys=mdlDerivatives(t,x,u);
    case {2,4,9},
      sys = [];
    otherwise
      error(['Unhandled flag = ', num2str(flag)]);
    end
\end{verbatim}
\texttt{Знать, что такое динамическое моделирование}

\paragraph{Пример}\mbox{}\par
Создадим модель с умножением сигнала на заданный коэффициент
%\begin{figure}
% \centering
% \begin{tikzpicture}[node distance = 2cm]
%   \node[block] (const) {1 (const)};
%   %\node[block, right of = const] (sf) {S\_function\_1}
%   %\node[block, right of = sf] (d) {display};
%   %\draw[arrow] (const) -- (sf);
%   %\draw[arrow] (sf) -- (d);
% \end{tikzpicture}
%\end{figure}

Код:
\begin{verbatim}
  %s_function_1.m 
  function[sys, x0, str, ts] = s_function_1(t, x, u, flag, k)
  switch flag,
    case 3,
      sys=mdlOutpust(t, x, u, k)
    case {1, 2, 4, 9},
      sys = [  ];
    case 0,
      sys = mdlInitializeSizes

  %----------------------------
  %mdlInitializeSizes
  ...
  sizes = simsizes;
  sizes.NumContStates = 0;
  sizes.NumDiscStates = 0;
  sizes.NumOutputs = -1;
  sizes.NumInputs = -1;
  sizes.DirFeedThrough = 1;
  sizes.SampleTimes = 1;

  sys = simsizes(sizes);
  x0 = [];
  str = [];
  ts = [0 0]

  %-------------------------
  %mdlOutputs
  function sys = mdlOutputs(t, x, u, k)
  sys = u * k;

\end{verbatim}

\paragraph{Пример}\mbox{}\par
\[
  \left\{\begin{aligned} 
    \dot X &= AX + BU \\
    Y &= CX + DU \\
  \end{aligned}\right. 
\] 
$U, X, Y$ имеют конкретную размерность равную двум.\\
$A, B, C, D$ Имеют размерность $2 \times 2$ \\
на вход подключен блок const со значением [1 2].
\begin{figure}
  \centering
  \begin{tikzpicture}[node distance = 3.5cm]
    \node[block] (const) {$[1\ 2]$ (const)};
    \node[block, right of = const] (sf) {S\_function\_2};
    \node[block, right of = sf] (d) {Display};
    \draw[arrow] (const) -- (sf);
    \draw[arrow] (sf) -- (d);
  \end{tikzpicture}
\end{figure}
\[
  A = \begin{bmatrix} 0.1 & 0.2 \\ -0.8 & 0.8 \end{bmatrix} \qquad
  B =\begin{bmatrix} 0.1 & 0 \\ 0.5 & 0 \end{bmatrix} 
\] 
\[
  C = \begin{bmatrix} 1 & 0 \\ 0 & 1  \end{bmatrix} \qquad
  D = \begin{bmatrix} 0.1 & 0 \\ 0 & -0.2 \end{bmatrix} 
\] 
Код:
\begin{verbatim}
  %s_function_2.m
  function[sys, x0, str, ts] = cont_common(t, x, u, flag, A, B, C, D)

  switch flag,
    case 0,
      sys = mdlInitializeSizes;
    case 1,
    sys = mdlDerivatives(t, x, u, A, B)
    case 3,
      sys = mdlOutputs(t, x, u, C, D);
    case {2, 4, 9},

  %----------------------------
  %mdlInitializeSizes
  ...
  sizes = simsizes;
  sizes.NumContStates = 2;
  sizes.NumDiscStates = 0;
  sizes.NumOutputs = 2;
  sizes.NumInputs = 2;
  sizes.DirFeedThrough = 1;
  sizes.SampleTimes = 1;

  sys = simsizes(sizes);
  x0 = [0 0];
  str = [];
  ts = [0 0]

  %----------------------------
  %mdlDerivatives
  function sys = mdlDerivatives(t, x, u, A, B)
  sys = A*x + B*u;

  %----------------------------
  %mdlOutputs
  function sys = mdlOutputs(t, x, u, C, D)
  sys = C*x + D*u;
\end{verbatim}
В блоке constant:\\
Constant value: $[1\ 2]$\\
--------------\\
name: s\_function\_2
parameters: $\begin{bmatrix}0.1 & 0.2; & -0.8 & 0.8 \end{bmatrix};
\begin{bmatrix} 0.1 & 0; & 0.5 & 0 \end{bmatrix};$ 
