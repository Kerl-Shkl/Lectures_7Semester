%! TEX root = ../main.tex
\usepackage{amsmath}
\section{Лекция от 10.9.2022. Моделирование систем управления.}
\subsection{Современные требования ПО моделирования.}
Интенсивное развитие средств визуализации и ООП, а также соответствующего
программного обеспечения позволяет программистам создавать среду в рамках
которой:
\begin{enumerate}
  \item Предлагается исчерпывающий набор инструметнов, который позволяет любому
    пользователю самостоятельо создать адекватный визуальный образ модели
    объекта в формально информационном формате с навязываемым указанием всех
    необходимых входных данных.
  \item Ведется автоматический контроль как отдельных компонентов, так и всей
    модели с проверкой корректности и полноценности вводимых данных, что
    позволяет при любой квалификации пользователя принимать к расчету только
    заведомо корректные и полноценные модели.
  \item Обеспечивается дружественная информационная поддержка пользователя с
    описанием технологии моделирования, например, во встроенной справке, выдачей
    сообщений об ошибке, их описанием и указанием способа их устранения, выводом
    соответствующих подсказок.\\{\ttfamily Справка для 3-й работы в формате
    .chm. Инструмент - [MS HTML Workshop]}
  \item Автоматически формируется и расчитывается математическая модель на
    основе данной формально информационной модели.
  \item На основе результатов расчета математической модели автоматически
    воссоздаются основные и сопутствующие выходные характеристики.
  \item Выполняется визуализация выходных характеристик в
    количественном (таблица) и качественном (график) формате.
\end{enumerate}\par
Реализация пунктов 1--3 относится к среде создания и редактирования визуальных
образов модели (т.е. препроцессору). Пункт 4 относится к расчетному модулю (т.е.
процессору). Пункты 5--6 относятся к среде визуализации результатов расчета и
выходных характеристик (т.е. постпроцессору).\par

\subsection{Что такое алгоритм?}
\textbf{Алгоритм} --- это конечный набор правил однозначно раскрывающих
содержание и последовательность выполнения операций для решения поставленной
задачи.\par
\textbf{Алгоритмическая модель} и \textbf{алгоритмический процесс}. Свойства:
\begin{enumerate}
  \item Дискретность. Четкая последовательность отдельных действий. Действия
    должны быть связаны причинно следственными связями.
  \item Понятность и однозначность (детерменированность). Все действия и их
    порядок должны быть однозначно понятны. Возможность неоднозначного трактования должна быть исключена. Последовательность и содержание операций в алгоритмическом процессе
    должны полностью соответствовать описанию в алгоритмической моделе.
  \item Результативность или конечность. Алгоритм должен выполняться за конечное
    число шагов, каждый маршрут должен заканчиваться выводом результата.
  \item Адекватность и правильность. Алгоритм должен адекватно имитировать
    реальный процесс.
  \item Массовость. Алгоритм должен быть применим ко всему множеству входных
    данных.
  \item Сложность и ресурсоемкость.\\
    {\ttfamily Ресурсоемкость - количественный показатель требуемой памяти и
    времени. 
    
    Сложность - порядок зависимотси времени и памяти необходимых для
    выполнения алгоритма от его размерности. T(N) - временная сложность; O(N) -
    пространственная сложность. 
    
    Сложность сама по себе не дает определить ни
    время, ни объем необходимые для работы алгоритма. А позволяет только лишь
    оценить увеличение этих показателей по мере роста размерности.}
\end{enumerate}

\subsection{Типы моделей}
Типы моделей:
\begin{enumerate}
  \item \textbf{Феноменологические} и \textbf{абстрактные}. Феноменологические
    привязаные к процессу (локальные). Абстрактные воспроизводят систему с
    учетом ее  внутренней структуры и расширяют класс решаемых задач.
  \item \textbf{Активные} и \textbf{пассивные}. Активные взаимодействуют с 
    пользователем.
  \item \textbf{Статические} и \textbf{динамические}. Статические описывают 
    процессы без развития.
  \item \textbf{Дискретные} и \textbf{непрерывные}. Дискретные изменяют 
    состояние состояние  переменных скачком.
  \item \textbf{Детерменированные} и \textbf{стохастичесие}. Если между входными
    воздействиямии, переменными состояния и выходными сигналами есть однозначное 
    соответствие  для каждого момента, то модель детерменированная. Если модель
    стохастическая, то применяется только статистический подход. И При этом
    изменения состояния объекта и выходных стостояний задается в виде
    вероятностных распределений.
  \item \textbf{Функциональные} и \textbf{объектные}. Если описание исходит из
    поведения системы, то модель построена по функциональному признаку. Если 
    описание объекта отделено от описания другого объекта и поведение системы
    вытекает из свойств другого объекта, то это объектная модель.
\end{enumerate}
{\ttfamily Класс - сочетание или множество объектов обладающих идентичными
свойствами (атрибутами) и эквивалентными операциями (методами).}
