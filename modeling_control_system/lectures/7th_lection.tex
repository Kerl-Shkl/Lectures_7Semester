%! TEX root = ../main.tex
\section{Лекция от 15.10.2022}
Алгебраический контур --- это такая часть цепи, в которой выходной сигнал блока
в последствии попадает на вход этого блока (зацикливание).\par
Блоки:
\begin{itemize}
  \item Осцилограф --- \textit{Scope}: параметры (\textit{min, max, label,
    sample time {\normalfont\ttfamily Если ($-1$) --- наследование шага
    квантования из предыдущей части системы единицы измерения.}, limit data
    prints to last)}. Signal and ports --- количество портов.
  \item Цифровой дисплей --- \textit{display}: параметры --- типы
    (\textit{long}) и т.д.)
  \item \textit{Floating Scope} --- множество сигналов в одном месте.
  \item \textit{Model Lineariser} (переходная характеристика, годограф
    Найквиста) --- время регулирования, перерегулирования.
\end{itemize}

