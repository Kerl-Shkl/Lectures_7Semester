%! TEX root = ../main.tex
\section{Лекция от 24.09.22}
\subsection{Математическое моделирование в составе системного исследования}
Для решения сложных проблем обычно применяют системное исследование, которое
включает в себя следующие этапы решения проблемы:
\begin{enumerate}
	\item Изучение предметной области.
	\item Выявление и формулирование проблемы.
	\item Математическая (формальная) постановка проблемы (формализация
		постановки).\\ {\ttfamily  Надо понять какие алгоритмы необходимы}
	\item Математическое моделирование исследуемых объектов и процессов.
	\item Статистическая обработка результатов моделирования.
	\item Формулирование альтернативных решений.
	\item Оценка альтернативных решений.
	\item Формулирование выводов и предложений по решению проблемы.
	\item Математическая модель, как результат системного исследования, в
		обобщенном дискретном по времени виде представляет собой набор
		соотношений:\\
		\begin{align*}
			Y(t_n) = F[U(t_n), X(t_n)] \\
			X(t_{n+1}) = G[U(t_n), X(t_n)] \\
			U(t_{n+1}) = H[U(t_n), X(t_n)]\\
		\end{align*}
		$U(t_n), U(t_{n+1})$ --- множествео входных параметров в
		смежные моменты  времени.\\ $X(t_n), X(t_{n+1})$ --- множество переменных
		состояния в смежные моменты времени.\\
		$Y(t_n)$ --- Множество выходных параметров.
\end{enumerate}
\begin{center}
	\begin{tikzpicture}[auto, node distance = 2cm, >-latex']
		\node [input, name=input] {};
		\node (block) [block, right of=input] {X};
		\node (out) [output, right of=block] {};
		\draw [->] (input) -- node {$U$} (block);
		\draw [->] (block) -- node {$Y$} (out);
	\end{tikzpicture}
\end{center}
\begin{gather*}
	X(t_{n+1}) = AX(t_n) + BU(t_n) \\
	Y(t_n) = CX(t_n) + DU(t_n)\\
	\dot{X} = AX + BU \\
	Y = CX + DV
\end{gather*}

\subsection{Проектирование математической модели}
Для составления математической модели в одной их описаных форм необходимо:
\begin{enumerate}
	\item Описание целевой функции системы с обозначением входных и выходных
		величин, а также показатель ее качества.
	\item Описание конфигурации (состава) системы с представлением ее компонентов,
		их взаимосвязи, придаваемых величин, а также граничных и начальных условий,
		в которые поставлены компоненты и вся система в целом.
	\item Определение физического характера каждого из компонентов или всей
		системы в целом, если все ее компоненты идентичны с физической точки зрения
		по протекающим в них процессам.
	\item Изыскания для компонентов или всей системы в целом тех уравнений,
		которые в соответствии с определенным в пункте 3 физическим характером
		обобщенно или локально, но обязательно адекватно и полноценно описывает
		преобразование входных величин в выходные.
	\item Локализация (конкретизация) уравнений состояния каждого компонента
		системы или всей системы в целом с учетом конфигурации описанной в пункте 2
		и при необходимости переход от дифференциальной формы этих уравнений к
		матричной. Кроме того модель должна быть представительной в контексте
		адекватной передачи нужных свойств объекта, а также необходимо совпадение
		выходных параметров модели и объекта с определенной точностью. Точность
		конечного результата зависит как, от изначальной погрешности математической
		модели связанной с применяемыми методами, так и от погрешности, которую
		вносят алгоритмическая модель и алгоритмический процесс.
\end{enumerate}\par
Таким образом, математическая модель должна быть спроектирована так, чтобы
соответствующая алгоритмическая модель выдержала определенные требования к
показателям качества. Оптимальная математическая модель предобуславливает
возможность своей алгоритмизации и реализации на ЭВМ с минимально погрешность и
минимальными затратами на разработку.
