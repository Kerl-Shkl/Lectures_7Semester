%! TEX root = ../main.tex
\section{Лекция от 8.10.22}
\subsection{Переход от передаточной функции к переменным состояния}
Передаточная функция в общем виде.\par
\[W(p) = \frac{b_mp^m + b_{m-1}p^{m-1} + \ldots + b_1p + b_0}
{a_np^n + a_{n-1}p^{n-1} + \ldots + a_1p +a_0} = \frac{(2)}{1}\]
\begin{enumerate}
	\item Метод фазовых переменных или метод прямого программирования.
		\[\frac{U(p)}{(1)} = \frac{Y(p)}{(2)} = X(p)\]
		\[\left\{\begin{aligned} 
				&(1)\cdot X(p) = U(p)\\
				&(2) \cdot X(p) = Y(p)
				    \end{aligned}\right| p^nX(p) = \frac{1}{a_n}(U(p) - a_{n-1}p^{n-1}X(p) - \ldots - a_1pX(p) - a_0X(p))\]
		\[x^{(n)} = \frac{1}{a_n}(u - a_{n-1}x^{(n-1)} - \ldots - a_1 \dot x - a_0x)\]
		\[x_1 = x, x_2 = \dot x, x \text{не успел}\]
		\[\begin{aligned} 
			&\dot x_1 = x_2\\
			&\dot x_2 = x_3\\
			&\vdots\\
			&\dot x_{n-1} = x_n\\
			&\dot x_n = \frac{1}{a_n}(u - a_{n-1}x_n - \ldots - a_1x_2 - a_0x_1)\\
			\hline\\
			&y =  b_mx^{(m)} + b_{m-1}x^(m-1) + \ldots + b_1\dot x + b_0x
		\end{aligned}\]

		\[\begin{bmatrix} \dot x_1 \\ \dot x_2 \\ \dot x_3 \\ \vdots \\ \dot x_{n-1}
		\\ \dot x_n \end{bmatrix} = 
		\begin{bmatrix} 0 & 1 & 0 & \hdots & 0 & 0 \\
										0 & 0 & 1 & \hdots & 0 & 0 \\
										0 & 0 & 0 & \hdots & 0 & 0 \\
										\vdots & \vdots & \vdots & \hdots & \vdots & \vdots \\
										0 & 0 & 0 & \hdots & 0 & 1\\
										-\frac{a_0}{a_n} & -\frac{a_1}{a_n} & -\frac{-a_2}{a_n} &
										\hdots & -\frac{a_{n-1}}{a_n} & -\frac{a_{n-1}}{a_n} 
										\end{bmatrix}	\begin{bmatrix} 
		x_1 \\ x_2 \\ x_3 \\ \vdots \\ x_{n-1} \\ x_n
										\end{bmatrix} + \begin{bmatrix}
									0 \\ 0 \\ 0 \\ \vdots \\ 0 \\ \frac{1}{a_n} 
									\end{bmatrix} \cdot u
		\]

		\[\dot{\bar X} = A \bar X + B \bar U\]

	\item Метод полюсов или параллельного программирования.
		При отсутствии кратных полюсов передаточная функция раскладывается на дроби
		типа $\frac{A_i}{p - \lambda_i}$. $A_i$ --- вычеты; $\lambda_i$ --- полюса.
		\[A_i = (p - \lambda_i)W(p) |_{p=\lambda_i}\]
		\[X_i(p) = \frac{U(p)}{p-\lambda_i}, \]
		\[\dot X = \begin{bmatrix} \lambda_1 & 0 & 0 & \hdots & 0 \\
			0 & \lambda_2 & 0 & \hdots & 0 \\
			0 & 0 & \lambda_3 & \hdots & 0 \\
			\vdots & \vdots & \vdots & \hdots & \vdots\\
			0 & 0 & 0 & \hdots & \lambda_n
			\end{bmatrix} \cdot X + \begin{pmatrix}
			1 \\ 1\\ 1\\ \vdots 1
		\end{pmatrix} \cdot u\]
		\[X = \begin{pmatrix}x_1 \\ x_2 \\ x_3 \\ \vdots \\ x_n \end{pmatrix}\qquad Y = CX
			= \begin{pmatrix}A_1 & A_2 & \hdots & A_n\end{pmatrix} \cdot X\]
	\end{enumerate}\par

Рассмотрим пример:
\[\ddot y + 3 \dot y + 2y = u + 4 \dot u\]
\[W(p) = \frac{Y(p)}{U(p)} = \frac{1 + 4p}{p^2 + 3p + 2} = \frac{7}{p+2} -
\frac{3}{p+1}\]
\[\left.\begin{aligned} &\dot x_1 = -2x_1 + u \\
									&\dot x_2 = -x_2 + u \\
									&y = 7x_1 - 3x_2\end{aligned}\right\} 
\left\{\begin{aligned}
	&\dot x = \begin{pmatrix} -2 & 0 \\ 0 & -1\end{pmatrix}x +
	\begin{pmatrix}1 \\ 1\end{pmatrix}u\\
	&y = (7 - 3)x
\end{aligned}\right.\]
\paragraph{Кратные подвещественные полюса}\mbox{}\par
\[W(p) = \frac{A_1}{(p-\lambda_i)^k} + \frac{A_2}{(p-\lambda_i)^{k-1}} + \ldots
+ \frac{A_k}{p - \lambda_i} + \frac{A_{k+1}}{p - \lambda_{k+1}} + \ldots +
\frac{A_n}{p-\lambda_n}\]
Где $\lambda_i$ --- кратности $k$

\[\left\{\begin{aligned} &x_k(p) = \frac{U(p)}{p-\lambda_i}\\
												 &x_{k-1}(p) = \frac{x_k(p)}{p-\lambda_i}\\
												 &\hdots\\
												 &x_1(p) = \frac{x_2(p)}{p-\lambda_i}
	\end{aligned}\right. \Longrightarrow \left\{\begin{aligned}
												 &\dot x_k = \lambda_i x_k + u \\
												 &\dot x_{k-1} = \lambda_i x_{k-1} + x_k\\
												 &\hdots\\
												 &\dot x_1 = \lambda_1 x_1 + x_2
\end{aligned} \right.\]

		\[\dot X = \begin{bmatrix} \lambda_i & 1 & 0 \hdots \\
			0 & \lambda_i & 1 & \hdots \\
			\vdots & 0 & \lambda_i & 1 & \hdots\\
			\vdots & \vdots & \ddots & \ddots & \ddots &\hdots\\
			\vdots & \vdots & \vdots & 0 & \lambda_i & 1 & \hdots\\
			\vdots & \vdots & \vdots & \vdots & 0 & \lambda_i & 0 & \hdots\\
			\vdots & \vdots & \vdots & \vdots & \vdots & 0 & \lambda_{k+1} & 0
						 &\hdots\\
			\vdots & \vdots & \vdots & \vdots & \vdots & \vdots & \vdots & \ddots
						 &\hdots\\
			\vdots & \vdots & \vdots & \vdots & \vdots & \vdots & \vdots & \vdots &
			\lambda_n
			\end{bmatrix} \cdot x + \begin{pmatrix} 0 \\ 0 \\ 0 \\ \vdots\\ 0\\ 1\\
			1\\ \vdots \\ 1
		\end{pmatrix} \cdot u\]

Пример 2:
\[W(p) = \frac{p}{(p+1)^2(p+2)} = - \frac{1}{(p+1)^2} + \frac{2}{p+1} -
\frac{2}{p+2}\]
\null
\[A_1 = \left.\frac{(p+1)^2p}{(p+1)^2(p+2)}\right|_{p=-1} = \frac{-1}{-1+2} = -1\]
\null
\[A_2 = \left.\frac{(p+2)p}{(p+1)^2(p+2)}\right|_{p=-2} = \frac{-2}{(-2+1)^2}
= -2\]
\parТак как сумма двух вычетов ($A_2$ и $A_3$) в двухкратном полюсе равна 0, то
$A_3 = 2$.
\[\left\{\begin{aligned} 
&\dot X = \begin{pmatrix} -1 & 1 & 0\\
	0 & -1 & 0\\
	0 & 0 & -2
	\end{pmatrix} \cdot X + \begin{pmatrix} 0 \\ 1 \\ 1 \end{pmatrix} \cdot u\\
		&Y = \begin{pmatrix}-1 & 2 & -2\end{pmatrix} x
\end{aligned}\right.\]
%\hrule{0.5pt}{\textwidth}
-----------------------------------------
\null
\parПри комплексно-сопряженных полюсах есть возможность сформировать матрицы
состояния только с вещественными элементами.
\[\left\{\begin{aligned} &\dot x = Ax +Bu\\
		& y = Сx
	\end{aligned}\right.\]
		\[A = \begin{pmatrix} a + jb & 0 \\ 0 & a - jb \end{pmatrix},
		B = \begin{pmatrix} 1 \\ 1 \end{pmatrix}\]
Введем переменные:
\[Z = MX,\quad X = M^{-1}Z\]

При наличии комплексно сопряженных полюсов есть возможность составить матрицы
состояний только с вещественными элементами.

\[\left\{\begin{aligned}&Z = MAM^{-1} \cdot Z + MBu\\
												& Y = CM^{-1}Z\end{aligned}\right. \qquad
		M = \begin{pmatrix} 1 & -1 \\ j & 1 \end{pmatrix}
\]

\[\left\{\begin{aligned} & Z = \begin{pmatrix} a & b \\ -b & a \end{pmatrix} Z + \begin{pmatrix} 2 \\ 0 \end{pmatrix} u \\
												 & Y = CM^{-1}\cdot Z
\end{aligned}\right.\]

\subsection{Переход от уравнений состояний к передаточной функции}
\[
  \left\{\begin{aligned} & \dot X = AX + BU \\
  & Y = CX + DU\end{aligned}\right. 
\] 

\[
  \left\{\begin{aligned} & pX = AX + BU
  & Y = CX + DU \end{aligned}\right.  \rightarrow \left\{\begin{aligned} & X =
  (pE \cdot A)^{-1}BU \\ & Y = (C(pE \cdot A)^{-1}B +D)\cdot U \end{aligned}\right. 
\] 
\[
W(p) = C(pE \cdot A)^{-1}B + D
\] 
Пример:
\[
\begin{aligned}
   \dot X &=  \begin{bmatrix} -1 & -2 \\ 0 & -2 \end{bmatrix} X + \begin{pmatrix}
1 \\ 1\end{pmatrix} U \\
   Y &= \begin{pmatrix} -1 & 1 \end{pmatrix} X
\end{aligned}
\quad \Rightarrow \quad
\begin{aligned}
 pX &= \begin{bmatrix} -1 & -2 \\ 0 & -2 \end{bmatrix} \bar X + \begin{pmatrix}
 1 \\ 1 \end{pmatrix} U \\
 Y &= \begin{pmatrix} -1 & 1 \end{pmatrix} \bar X
\end{aligned}
\]
\bigskip
\[
  pE\cdot A = \begin{pmatrix} p+1 & 2 \\ 0 & p+2 \end{pmatrix} \qquad (pE\cdot
  A)^{-1} = \frac{1}{(p+1)(p+2)}\begin{pmatrix} p+2 & -2 \\ 0 & p+1 \end{pmatrix}
\] 
\bigskip
\[
  X = (pE \cdot A)^{-1}BU = \frac{1}{(p+1)(p+2)}\begin{pmatrix} p+2 & -2 \\ 0 &
  p+1 \end{pmatrix} \begin{pmatrix} 1 \\ 1 \end{pmatrix} U =
  \frac{1}{(p+1)(p+2)} \begin{pmatrix} p \\ p+1 \end{pmatrix} U
\] 
\bigskip
\[
Y = \frac{U}{(p+1)(p+2)} \qquad W(p) = \frac{1}{(p+1)(p+2)}
\] 

