%! TEX root = ../main.tex

\section{Лекция от 26.11.2022}
\subsection{Повышение эффективности моделирования в среде Matlab}
\subsubsection{Способ ускорения расчетов модели}
Скорость рвсчета модели можно увеличить следующими способами:
\begin{enumerate}
  \item Перейти от обычного режима к ускоренному. ($\text{Normal} \to
    \text{Accelerator}$). В этом режиме предворительно компилируется модель в
    формат DLL. 
  \item Заменить блоки memory на блоки unitDelay. Если для расчета модели
    применяется метод расчета с переменным шагом (variableStep) и переменным
    порядком есть блок memmory, то на каждом расчетном шаге порядок снижается
    до 1 (это снижает скорость расчета).
  \item Заменить непрерывную модель на дискретную.
  \item Увеличить максимальное время расчета (maxStepSize).
  \item Сузить расчетный интервал.
  \item Исключить алгебраические контуры.
  \item Снизить количество блоков Scope. LimitDataPointsToLast должен быть
    больше, чем количество расчетных точек.
  \item Ввести прореживание (Decimation([12]))
\end{enumerate}

\subsubsection{Способы уточнения расчетов модели}
Существует три классических способа:
\begin{enumerate}
  \item При значительном отличии выходных данных в начале рассчетного интервала
    следует в явном виде задать достаточно малый начальный шаг расчета.
  \item Попробовать изменть погрешности Relative tolerance и Absolute tolerance.
  \item 
\end{enumerate}

\subsubsection{Профилеровщик (Profiler)}

\subsubsection{Синтез систем управления}
зф
