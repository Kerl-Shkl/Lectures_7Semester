%! TEX root = ../main.tex

\section{Лекция от 22.10.2022}
\subsection{Моделирование систем управления на командном языке \textit{Control
System Toolbox (CST)}}
Команды:
\begin{enumerate}
  \item \textit{zpk} --- создает звено с явно определенной передаточной функцией.
    \[
      W(s) = k \frac{(s-z_1)(s-z_2)\ldots(s-z_n)}{(s-p_1)(s-p_2)\ldots(s-p_n)}
      \qquad \text{zpk форма}
    \] 
    \par Код:
    \begin{verbatim}
      obj = zpk(z, p, k)
      zpk([1 2], [3, 4, 5], 6)
    \end{verbatim}
    Вывод:
    \[
      W(s) = 6 \frac{(s-1)(s-2)}{(s-3)(s-4)(s-5)}
    \] 
  \item \textit{tf} --- создает звено tf формы.
    \[
      W(s) = \frac{N_1 s^{m} + N_2 s^{m-1} + \cdots + N_m s + N_{m+1}}{D_1 s^{s}
      + D_2 s^{n-1} + \cdots + D_ns + D_{n+1}}
    \] \par
    Код:
    \begin{verbatim}
      tf([1, 2], [3, 4, 5])
    \end{verbatim}
    Вывод:
    \[
      W = \frac{s+2}{3s^2 + 4s + 5} 
    \] 
  \item \textit{ss}
    \[
      \left\{\begin{aligned} 
        \dot X &= AX + BU\\
        Y &= CX + DU
      \end{aligned}\right. 
    \] 
    Код:
    \begin{verbatim}
      obj = ss(A, B, C, D) 
    \end{verbatim}

  \item \textit{linmod}
    \begin{verbatim}
      [z, p, k] = linmod('name')
      [n, d] = linmod('name')
      [a, b, c, d] = linmod('name')
    \end{verbatim}
    Где \verb!'name'! --- путь до модели из симулинка.

  \item \textit{parallel} (идентичная ей \textit{plus})
    \begin{verbatim}
      obj = parallel(obj1, obj2) 
    \end{verbatim}
    \begin{tikzpicture}[auto, node distance=2cm]
     \node[input] (in) {};
     \node[block, above right of = in, xshift=1cm] (o1) {Obj1};
     \node[block, below right of = in, xshift=1cm] (o2) {Obj2};
     \node[sum, right of = in, xshift=3.5cm] (sum) {};
     \node[output, right of = sum, xshift = -1cm] (out) {};
     \draw[arrow] (in) -- +(0.5, 0) |- (o1);
     \draw[arrow] (in) -- +(0.5, 0) |- (o2);
     \draw[arrow] (o1) -| (sum);
     \draw[arrow] (o2) -| (sum);
     \draw[arrow] (sum) -- (out);
     %\draw[arrow] (in) -- (o2);
    \end{tikzpicture}\par
    %TODO вставить картинку
    Расширенная форма: parallel

  \item \textit{series} (идентичная mtimes(obj1, obj2))
    \begin{verbatim}
      obj = series(obj1, obj2) 
    \end{verbatim}

  \item \textit{feedback}
    \begin{verbatim}
     obj = feedback(obj1, obj2) //(obj1, obj2, -1) (obj1, obj2, +1)
    \end{verbatim}
    \begin{figure}[h!]
      \centering
      \begin{tikzpicture}[auto, node distance = 2cm]
        \node[input] (in) {};
        \node[sum, right of = input] (s) {};
        \node[block, right of = s] (o1) {$obj_1$};
        \node[output, right of = o1, xshift=1cm] (out) {};
        \node[block, below of = o1] (o2) {$obj_2$};
        \draw[arrow] (o1) -- node(p){y} (out);
        \draw[arrow] (in) -- (s);
        \draw[arrow] (s) -- (o1);
        \draw[arrow] (p) |- (o2);
        \draw[arrow] (o2) -| (s);
      \end{tikzpicture}
    \end{figure}

      \item \textit{\dots data}
    \begin{verbatim}
      [z, p, k] = zpkdata(obj)
      [num, den] = tfdata(obj)
      [a, b, c, d] = ssdata(obj)  
      Пример:
      obj = zpk([1, 2], [3 4 5], 6)
      Выдаст:
      num = [1 x 4 double] | num{1} -> ans = 0 6 -18 -12
      den = [1 x 4 double] | den{1} -> ans = 1 -12 47 -60
    \end{verbatim}
    В фигурных скобка указывается индекс интересующей передаточной функции в
    матричной структуре представления параметров объекта. Эта позиция задается
    либо одним числом для скалярных объектов (1 вход 1 выход) и векторных (1
    вход много выходов), либо парой чисел для матричных объектов.\par

    Для вывода определенного элемента строки коэфициентов числителя или
    знаменатиеля, нулей или полюсов после указания в фигурных скобках индекса
    соответствующей передаточной функции надо дополнительно в круглых скобках
    задать позицию этого элемента в строке.
\end{enumerate}

\paragraph{Пример:}
\begin{verbatim}
  H = [tf([1 2], [3, 4, 5]), tf([1], [2 3])]
  H.den{1}(3) = 7 //Изменим тертий элемент
\end{verbatim}
Вывод:\\
\verb!Transfer function from input1 to output!
\[ \frac{s+2}{3s^2 + 4s + 7} \] 
\verb!Transfer function from input2 to output!
\[ \frac{1}{2s + 3} \] 
\texttt{InputName --- имена входов;\\ OutputName --- имена выходов в формате
одномерного массива; \\ Variable --- форма отображения переменной преобразования
Лапласа;\\ set(obj) --- позволяет получить информацию о формате представления
свойств объекта;\\ get(obj) --- }\par
\begin{verbatim}
  obj = tf(1, [2, 3], 'inputname', 'вход', 'outputname', 'выход', 'variable', 'p')
  set(obj, 'inputname', 'arseniy', 'outputname', 'отлично за колоквиум',
  'variable', 's') | get
\end{verbatim}

\subsubsection{Анализ CST модели}
\begin{itemize}
  \item step(obj) --- реакция системы на единичный воздействующий сигнал.
  \item impulse(  ) --- реакиция на импульс.
  \item bode --- строит ЛАЧХ и ФЧХ.
  \item nyquist --- годграф Найквиста.
  \item margin --- 
  \item pole --- выводит полюса.
  \item zero --- выводит нули.
  \item pzmap --- выводит полюса и нули.
  \item \verb![y, t, x] = step(obj)!
  \item \verb![mag, ph, freq] = bode(obj) #| nyquist(obj)!
  \item \verb![ma, mp, fma, fmp] = margin(obj)! --- запасы устойчивости.
  \item \verb![pole, zero] = pzmap(obj)!
\end{itemize}\par
Кроме того эти команды можно использовать для одновременного вывода в одном окне
расчетных характеристик нескольких объектов, для чего эти объекты надо просто
перечислить через запятую в круглых скобках.
\paragraph{Пример:}
\begin{verbatim}
  obj1 = tf([1 2], [3 4 5]);
  obj2 = tf([6 7], [8 9 10]);
  pzmap(obj1, obj2)
\end{verbatim}
