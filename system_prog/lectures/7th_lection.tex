%! TEX root = ../main.tex

\section{Лекция от 10.12.2022}

\subsection{Синхронизация задач с помощью функций ожидания}
Освобождение объекта --- переход в сигнальное состояние.
\begin{verbatim}
  DWORD WaitForSingleObject(
                HANDLE hHandle,
                DWORD dwMilliseconds
  );
\end{verbatim}
Второй параметр часто задают как \verb!INFINITE!.

Функция может вернуть одно из четырех значений:
\begin{itemize}
  \item \verb!WAIT_TIMEOUT! --- Время ожидания вышло, объект не освободился.
  \item \verb!WAIT_OBJECT_0! --- Объект перешел в сигнальное состояние.
  \item \verb!WAIT_FAILED! --- Ошибка выполнения функции.
  \item \verb!WAIT_ABANDONED! --- только для мьютексов. Означает, что мьютекс
    освободился в следствии окончания выполнения владевшего им потока.
\end{itemize}

Функция:
\begin{verbatim}
  DWORD WaitForMultipleObjects(
                DWORD   nCount,
                HANDLE  *lpHandles,
                BOOL    bWaitAll,
                DWORD   dwMilliseconds
  );
\end{verbatim}
Если \verb!bWaitAll! установить в \verb!FALSE!, функция завершает свою работу,
когда хотя бы один из объектов переходит в сигнальное состояние. Если
\verb!bWaitAll! равен \verb!TRUE!, то функция ожидает освобождения всех
объектов.

Функция возвращает следующие значения:\par
Если \verb!bWaitAll == TRUE!:
\begin{itemize}
  \item \verb!WAIT_TIMEOUT! --- Время ожидания вышло, объект не освободился.
  \item \verb!WAIT_OBJECT_0! --- Объект перешел в сигнальное состояние.
  \item \verb!WAIT_FAILED! --- Ошибка выполнения функции.
  \item \verb!WAIT_ABANDONED! --- только для мьютексов. Означает, что мьютекс
    освободился в следствии окончания выполнения владевшего им потока.
\end{itemize}\par
Если \verb!bWaitAll == FALSE!:
\begin{itemize}
  \item \verb!WAIT_TIMEOUT!
  \item \verb!WAIT_FAILED!
  \item \verb![WAIT_OBJECT_0 ... WAIT_OBJECT_0 + nCount-1]!
  \item \verb![WAIT_ABANDONED_0 ... WAIT_ABANDONED_0 + nCount-1]!
\end{itemize}
Чтобы получить индекс объекта надо из вернувшегося значения вычесть
\verb!WAIT_OBWAIT_OBJECT_0!.

В Рихтере посмотреть функцию \verb!SignalObjectandWait!.

Одним из недостатков всех рассмотренных функций ожидания является то, что поток
в котором вызывается любая из этих функций останавливается и не воспринимает
поступающие сообщения. Сообщения поступают в очередь, но очередь не
разгружается. Решение этой проблемы путем установки малого значения
\verb!dwMilliseconds! и зацикливание вызова, если функция вернула
\verb!WAIT_TIMEOUT! помогает слабо, так как сообщения все равно не успевают
обработаться при циклической передачи управления.

Для решения этой проблемы есть функция 
\begin{verbatim}
   DWORD MsgWaitForMultipleObjects(
                DWORD   nCount,
                HANDLE  *lpHandles,
                BOOL    bWaitAll,
                DWORD   dwMilliseconds,
                DWORD   dwWakeMask
  );
\end{verbatim} 
Эта функция завершает свою работу не только при наступлении сигнальных состояний
объектов или по истечению времени ожидания, но и при поступлении сообщений,
определенных маской \verb!dwWakeMask!. Примеры масок:
\begin{itemize}
  \item \verb!QS_ALLPOSTMESSAGE! --- функция будет реагировать на асинхронные
    сообщения отличные от событий аппаратного ввода.
  \item \verb!QS_SENDMESSAGE! --- функция будет реагировать на синхронные
    сообщения, которые отправляются другим потоком.
\end{itemize}

\subsection{Синхронизация задач с помощью событий}
Событие --- Самые примитивные объекты ядра, а по сути своей простыми
уведомлениями об окончании каких-либо действий. Событие содержит в своем
дескрипторе содержит счетчик числа пользователей и две логические переменные,
которые указывают тип события (с ручным сбросом и с автоматическим сбросом) и
состояние (свободен или занят).

События с ручным сбросом можно перевести в несигнальное состояние функцией
\verb!ResetEvent!. События с автосбросом переводятся в несигнальное состояние
функцией \verb!ResetEvent! или функцией \verb!WaitForSingleObject!. Если событие
с автоматическим сбросом ожидает несколько потоков с помощью функции
\verb!WaitFoWaitForSingleObject!, то из состояния ожидания особождается только
один из этих потоков.

\begin{verbatim}
  HANDLE CreateEvent(
            LPSECURITY_ATTRIBUTES lpEventAttributes,
            BOOL                  bManualReset,
            BOOL                  bInitialState,
            LPCSTR                Name
  )
\end{verbatim}
Первый параметр объединяет в себе два параметра: наследование возвращенного
дескриптора дочерними процессами и структура безопасности дескриптора.
\verb!bManualReset! устанавливает тип сброса. Если \verb!TRUE!, то ручной,
\verb!FALSE! --- автоматический. \verb!bInitialState! устанавливает начальное
состояние события, \verb!TRUE! --- сигнальное, \verb!FALSE! --- несигнальное.
Чтобы проверить, что такое событие уже есть, надо выполниться следующий код:
\begin{verbatim}
  if (GetLastError() == ERROR_ALREADY_EXISTS) {...}
\end{verbatim}

Устанавливают состояния таймера следующий функции
\begin{verbatim}
  BOOL ResetEvent(HANDLE hEvent);
  BOOL SetEvent(HANDLE hEvent);
\end{verbatim}

Функция 
\begin{verbatim}
  BOOL PulseEvent(HANDLE hEvent)
\end{verbatim}
Если событие с ручным сбросом, эта функция позволяет всем потокам, которые
ожидают этого события и могут немедленно завершить ожидание, выйти из состояния
ожидания. Затем функция \verb!PulseEvent! переводи событие в несигнальное
состояние и завершает свою работу.

Для событий с автоматическим сбросом позволит только одному из ожидающих потоков
выйти из состояния ожидания, если это возможно, после чего переводит событие в
несигнальное состояние и завершает свою работу.

Функция 
\begin{verbatim}
  HANDLE OpenEvent(
          DWORD dwDesiredAccess,
          BOOL bInheritHandle,
          LPCSTR name
  );
\end{verbatim}
\verb!dwDesiredAccess! может быть установлен в трех вариантах:
\begin{itemize}
  \item \verb!EVENT_ALL_ACCESS! --- поток сможет применять к событию любые
    действия.
  \item \verb!EVENT_MODIFY_STATE! --- в потоке можно использовать функции только
    \verb!SetEvent! и \verb!ResetEvent!.
  \item \verb!SYNCHRONIZE! --- в потоке можно только использовать функции
    ожидания.
\end{itemize}

\paragraph{Пример}\mbox{}\par
Поток 1 готовит данные, которые должны быть обработаны потоками 2 и 3.
В потоке 1 размещаем следующий код:
\begin{verbatim}
  HANDLE E[2];
  E[0] = CreateEvent(NULL, TRUE, FALSE, "MyEvent2");
  E[1] = CreateEvent(NULL, TRUE, FALSE, "MyEvent3");
  HANDLE H = CreateEvent(NULL, TRUE, FALSE, "MyEvent1");
  // Подготовка данных
  ...
  // Окончание подготовки
  SetEvent(H);
  WaitForMultipleObjects(2, E, TRUE, INFINITE);
  ResetEvent(E[0]);
  ResetEvent(E[1]);
  ResetEvent(H);
\end{verbatim}

С помощью события MyEvent1 поток 1 будет оповещать потоки 2 и 3 об окончании
подготовки данных. События из массива \verb!E! будут использоваться для
получения информации от потоков 2 и 3. После подготовки вызываем функцию
\verb!SetEvent!.

В потоке 2:
\begin{verbatim}
  HANDLE H = CreateEvent(NULL, TRUE, FALSE, "MyEvent1");
  WaitForSingleObject(H, INFINITE);
  // Обработка данных
  ...
  // Окончание обработки
  SetEvent(OpenEvent(EVENT_ALL_ACCESS, TRUE, "MyEvent2"));
\end{verbatim}
