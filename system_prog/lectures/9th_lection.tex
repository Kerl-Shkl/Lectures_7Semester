%! TEX root = ../main.tex

\section{Лекция от 11.02.2023}
\subsection{Pipe}
\textbf{pipe} --- разделяемые участки памяти (псевдофайлы), которые используются
для передачи данных между разными потоками или процессами. Поток, который
создает канал, называется сервером. Потоки, которые подключаются к созданному
каналу, называются клиентами. Для передачи данных используются функции
\verb!ReadFile! и \verb!WriteFile!.
Каналы бывают:
\begin{itemize}
  \item односторонними (полудуплексными);
  \item двусторонними (дуплексными);
  \item анонимными (безымянные);
  \item именованные.
\end{itemize} 

Функция для создания канала:
\begin{verbatim}
  BOOL CreatePipe(
        PHANDLE hReadHandle,
        PHANDLE hWriteHandle,
        SECURITY_ATTRIBUTES AttrStr,
        DWORD dwSize 
        )
\end{verbatim}

Для соединения клиентского потока с анонимным каналом необходимо передать
клиенту один из дескрипторов анонимного пайпа. Передаваемый дескриптор должен
быть наследуемым. Наследование дескрипторов анонимного канала определяется
значением поля \verb!bInheritHandle! в структуре \verb!SECURITY_ATTRIBUTES!.
Если значение этого поля равно \verb!TRUE!, то дескрипторы создаваемого файла
создаются наследуемыми. Если дескрипторы анонимного канала создаются
наследуемыми, то тот дескриптор, который не передается клиенту, должен быть
сделан ненаследуемым и наоборот. Для этого есть функция \verb!DuplicateHandle!,
которая создает копию дескриптора, но у этой копии меняется наследование.
\begin{verbatim}
  BOOL DuplicateHandle(
          HANDLE hSourceProcessHandle,
          HANDLE hSourceHandle,
          HANDLE hTargetProcessHandle,
          PHANDLE hTargeHandle,
          DWORD dwDesiredAccess,
          BOOL bInheritHandle,
          DWORD dwOptions
  )
\end{verbatim}

Передача наследуемого дескриптора клиенту может выполняться одним из следующих
способов: 
\begin{enumerate}
  \item через командную строку;
  \item через поля (\verb!hStdInput hStdOutput hStdError!);
\end{enumerate}

При создании консольного (дочернего процесса) стандартные потоки ввода/вывода
связываются с дескрипторами, которые заданы в полях (пункт 2), для передачи
данных по анонимному каналу можно использовать функции стандартного
ввода/вывода. Такая процедура называется перенаправлением стандартного
ввода/вывода

Функции ввода вывода:
\begin{verbatim}
  BOOL WriteFile(
        _In_ HANDLE hAnonPipe,
        _In_ LPCVOID Buffer
        _In_ DWORD dwNumberOfButesToWrite,
        _Out_ LPDWORD lpNumberOfBytesWritten,
        _OptIn_ LPOVERLAPPED lpOverLap
  )
\end{verbatim}

\begin{verbatim}
  BOOL ReadFile(
        _In_ HANDLE hAnonPipe,
        _In_ LPCVOID Buffer
        _In_ DWORD dwNumberOfButesToRead,
        _Out_ LPDWORD lpNumberOfBytesReaded,
        _OptIn_ LPOVERLAPPED lpOverLap
  )
\end{verbatim}

Чтобы закрыть канал, нужно вызвать функцию \verb!CloseHandle(...)!

Порядок работы с анонимным каналом:\\
Сервер:
\begin{enumerate}
  \item Создает анонимный pipe с помощью функции \verb!CreatePipe!.
  \item Создает дубликат дескриптора, устанавливает нужное наследование с
    помощью функции \verb!DuplicateHandle!.
  \item Закрывает ненужный старый дескриптор.
  \item Подготавливает входные параметры для функции \verb!CreateProcess!.
  \item Создает дочерний процесс с помощью функции \verb!CreateProcess!, задавая
    \verb!bInheritHandle! как TRUE, и задавая флаг создания
    \verb!dwCreationFlags! как консольное приложение.
  \item Закрывает дескрипторы нового процесса.
  \item Выполняет чтение из анонимного канала.
  \item Закрывает дескриптор чтения из канала.
\end{enumerate}
Клиент:
\begin{enumerate}
  \item Записывает данные в канал.
  \item Закрывает дескриптор записи.
\end{enumerate}

\subsection{Синхронизация задач с помощью ожидаемых таймеров}
Ожидаемый таймер --- объект ядра, который самостоятельно переходит в сигнальное
состояние в определенное время или через регулярные промежутки времени. Чтобы
создать таймер необходимо вызвать функцию 
\begin{verbatim}
  HANDLE CreateWaitableTimer(
              LPSECURITY_ATTRIBUTES attributes,
              BOOL bManualReset,
              LPCSTR lpTimerName 
  )
\end{verbatim}
