%! TEX root = ../main.tex

\section{Лекция от 12.11.2022}

\dots

Системный вызов является частью интерфейса между операционной системой и
пользовательской программой. Пользовательская программа запрашивает сервис у
операционной системы, осуществляя системный вызов, и задача переходит в режим
ядра.

Кроме того есть исключения. Исключения --- события, возникающие в результате
выполнения программой недопустимой команды, доступу к ресурсу при отсутствии
ресурса или необходимых прав или обращению к отсутствующей странице памяти.
Исключения, как и системные вызовы, явялются синхронными событиями и делятся на
исправимые и неисправимые (фатальные).

Есть прерывания игнорируеммые и неигнорируемые (маскируемые и неигнорируемые).

Каждому прерыванию назначается свой уникальный приоритет.

\subsection{Диспетчеризация программ}
В рамках диспетчеризации выполняются следующие функции:
\begin{enumerate}
  \item Определение момента времени для смены исполняемой задачи.
  \item Выбор задачи из очереди готовых.
  \item Переключение контекстов задач. Выполняется на аппаратном уровне
    (привязано к платформе)
\end{enumerate}

{\bfАдаптивная диспетчеризация:}\\
Если задача слишком долго не запускается на выполнение, ее приоритет
уменьшается.\\
А если слишком долго выполняется, то ее приоритет уменьшается.

{\bf QNX:}
По истечении кванта времени ее приоритет уменьшается на единицу, если есть
другая задача с такимже приоритетом в готовом состоянии. Если задача с
пониженным приоритетом не выполняется в течении одной секунды, ее приоритет
повышается на единицу. Приоритет задачи не может превысить начальный уровень.
Если задача блокируется ей немедленно возвращается начальный приоритет.

Адаптивная диспетчеризация используется тогда, когда интенсивно выполняющиеся
фоновые процессы разделяют компьютер с пользовательскими процессами работающими
в диалоговом режиме.

\subsection{Проблема межзадачного взаимодействия}
Синхронизация задач заключается в согласовании времени их выполнения с помощью
приостановки до наступления некоторого события и последующей активизацией.  (в
третьей работе можно использовать критические секции (стр\, 218). Также засекать
время выполнения процесса (например GetProcessTime (172\,стр)), и каждого потока
(GetThreadTime)).

Синхронизация задач нужна, когда задачи взаимосвязанные, когда используются
разделяемые данные, когда необходимо синхронизировать потоки по времени.

Следующие функции:
\begin{itemize}
  \item \verb!WaitForSingleObject!
  \item \verb!WaitForMultipleObjects!
  \item \verb!CreateEvent!
  \item \verb!CreateMutex!
\end{itemize}

