%! TEX root = ../main.tex

\section{Лекция от 26.11.2022}
Важным объектом синхронизации является критическая секция программы, то есть ее
часть с доступом к разделяемым данным и непредсказуемым результатом выполнения
при параллельном изменении этих данных разными потоками.

Три основные проблемы:
\begin{enumerate}
  \item Гонки. 
  \item Тупики.
  \item Инверсия приоритетов. В системе есть три задачи. Задача n имеет низкий
    приоритет, задача v имеет высокий, а задача s низкий. Допустим активная
    задача n захватила ресурс r. Если задача v переходит в состояние готовности,
    она вытесняет задачу n, и ресурс r остается заблокированым. Решение:
    \begin{enumerate}
      \item Наследование приоритетов. Низкоприоритетная задача, захватившая
        ресурс наследует приоритет от высокоприоритетной задачи, которой этот
        ресурс нужен. Но в случае когда несколько средне и низкоприоритетных
        задач разделяют ресурсы с высокоприоритетной задачей, возможна ситуация
        когда высокоприоритетной задаче придется слишком долго ждать, пока
        каждая из задач с более низким приоритетом не освободит свой ресурс.
      \item Протокол предельного приоритета. К стандартным свойствам объекта
        синхронизации добавляется параметр, который равен максимальному
        приоритету задачи, которая обращается к этому объекту. Если этот
        параметр установлен, приоритет любой задачи повышается до указанного
        уровня и задача не сможет быть вытеснена никакой другой. После
        разблокирования ресурса приоритет задачи понижается до изначального
        уровня. Возможность задержки высокоприоритетных задач на время
        выполнения низкоприоритетных потоков.
    \end{enumerate}
\end{enumerate}

\subsection{Синхронизация задач на API уровне}

\subsubsection{Синхронизация в режиме пользователя}

\paragraph{Блокирующие переменные}\mbox{}\par
Каждому разделяемому ресурсу или набору критических данных назначается
глобальная двоичная переменная, которая равна 1, если ресурс свободен, и
значение 0, когда ресурс занят.

Специальные системные вызовы для работы с критическими секциями.
\begin{itemize}
  \item EnterCriticalSection
  \item LeaveCriticalSection
\end{itemize}

Порядок действий:
\begin{enumerate}
  \item Выделить объект типа CriticalSection
  \item Инициализировать объект критической секции функцией
    InitializeCriticalSection
  \item Перед входом в критическую секцию вызвать функцию EnterCriticalSection
  \item После завершения работу в критической секции вызвать
    LeaveCriticalSection
  \item После того, как критическая секция становится ненужной вызвать
    DeleteCriticalSection
\end{enumerate}

Можно войти в критическую секцию без блокировки потока функцией
TryEnterCriticalSection.

Время которое поток в очереди ожидает освобождения ресурса можно найти в реестре
по пути:
\begin{verbatim}
  HKEY_LOCAL_MACHINE\System\CurrentControlSet\Control\SessionManager ->
  CriticalSectionTimeout
\end{verbatim}
Фактически этот параметр составляет 30 суток
