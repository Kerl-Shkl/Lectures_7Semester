%! TEX root = ../main.tex
\section{Классификация операционных систем по функциональности}
\begin{enumerate}
	\item По количеству пользователей одновременно обслуживаемых системой
		операционные системы делятся на
		\begin{enumerate}
			\item Однопользовательские;
			\item Многопользовательские.
		\end{enumerate}
	\item По числу потоков, которые могут одновременно выполняться под управлением
		операционной системы
		\begin{enumerate}
			\item Однозадачные;
			\item Многозадачные.
		\end{enumerate}
\end{enumerate}\par
Наиболее характерными критериями эффективности вычислительных систем являются:
\begin{enumerate}
	\item Пропускная способность --- количество задач, выполняемых в единицу
		времени.
	\item Удобство работы пользователей, которое заключается в частности в том,
		что они имеют возможность интерактивного диалога одновременно с несколькими
		приложениями на одной машине.
	\item Реактивность системы, то есть способность системы выдерживать заранее
		заданные временные интервалы между запуском программы и получением
		результата.
\end{enumerate}
По критериям эффективности системы делятся на:
\begin{enumerate}
  \item Системы пакетной обработки. Главный критерий эффективности ---
    максимальная пропускная способность, то есть решение максимального
    количества задач в единицу времени. Используется следующая схема
    функционирования:\par
    В начале формируется пакет заданий $\to$ В момент начала
    работы системы из пакета формируется многозадачный пулл, одновременно
    выполняемых задач. При формировании пулла выбираются задачи, для выполнения
    которых используются разные ресурсы. При этом должна обеспечиваться
    сбалансированная загрузка всех устройств ЭВМ с минимальным простоем. Пока
    одна задача ожидает какого-либо события, процессор не простаивает.\par В
    системах под управлением пакетных операционных систем невозможно
    гарантировать выполнение конкретного задания в течении определенного периода
    времени. В системах пакетной обработки переключение процессора с одной
    задачи на другую выполняется по инициативе самой активной задачи. Поэтому
    существует высокая вероятность того, что одна задача может на долго занять
    процессор и выполнение интерактивных задач станет невозможным.\par
    Взаимодействие пользователя с вычислительной машиной, на которой 
    установлена пакетная операционная система сводится к тому, что пользоваетль 
    передает задание и ждет получения результата.

	\item Системы разделения времени. Каждому приложению попеременно периодически
    выделяется квант процессорного времени, по истечении которого переключаются
    на другое приложение (при его наличии). {\ttfamily POSIX требует наличие в
    системе двух политик \[ \left\{\begin{aligned} &FIFO \\ &RR
    \end{aligned}\right. \]}\\ Каждое приложение регулярно получает процессорное
    время. Пользователь регулярно получает возможность диалога с приложением. 
    Производительность операционной системы определяется скоростью смены
    контекста.\par
    Пропускная способность меньше, чем в пакетной обработке. Возрастают
    накладные расходы. \par
    Единственный критерий --- удобство и эффективность работы пользователя.

	\item Системы реального времени. Предназначены, как правило, для управления
    аппаратно-программными комплексами, для которых должны быть выполнены
    требования по их реактивности. Критерием эффективности является способность
    к выполнению критических задач в течении заданного интервала времени. Этот
    интервал называется временем реакции на событие (событие --- появление
    сигнала в операционной системе). Если задержка реакции системы на событие
    дольше определенного интервала недопустима, то система называется <<Системой
    жесткого реального времени>> (hard real time system). Иначе (если допустима)
    --- soft real time system.\par
    Многозадачный пулл представляет собой фиксированный набор заранее 
    разработанных программ, а выбор программы для выполнения осуществляется по 
    сигналам прерывания или в соответствии с расписанием плановых работ. \par
    Своппинг (подкачка) запрещены в системах реального времени.

\end{enumerate}
