%! TEX root = ../main.tex
\section{Классификация операционных систем по функциональности}
\begin{enumerate}
	\item По количеству пользователей одновременно обслуживаемых системой
		операционные системы делятся на
		\begin{enumerate}
			\item Однопользовательские;
			\item Многопользовательские.
		\end{enumerate}
	\item По числу потоков, которые могут одновременно выполняться под управлением
		операционной системы
		\begin{enumerate}
			\item Однозадачные;
			\item Многозадачные.
		\end{enumerate}
\end{enumerate}\par
Наиболее характерными критериями эффективности вычислительных систем являются:
\begin{enumerate}
	\item Пропускная способность --- количество задач, выполняемых в единицу
		времени.
	\item Удобство работы пользователей, которое заключается в частности в том,
		что они имеют возможность интерактивного диалога одновременно с несколькими
		приложениями на одной машине.
	\item Реактивность системы, то есть способность системы выдерживать заранее
		заданные временные интервалы между запуском программы и получением
		результата.
\end{enumerate}
По критериям эффективности системы делятся на:
\begin{enumerate}
	\item Системы пакетной обработки.
	\item Системы разделения времени.
	\item Системы реального времени.
\end{enumerate}
