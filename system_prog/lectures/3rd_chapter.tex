%! TEX root = ../main.tex
\section{Классификация операционных систем по архитектуре}
\subsection{Монолитное ядро}
Монолитное ядро --- архитектура операционной системы, при которой все ее
компоненты являются не самостоятельными модулями, а составными частями одной
программы, используют общие структуры данных и взаимодействуют друг с другом
путем непосредственного вызова процедур.
\subsection{Слоистые (многослойные) системы}
Вся система разбивается на ряд последовательных слоев с четко определенными
связями между ними, с тем, чтобы объекты слоя $n$ могли вызывать только объекты
слоя $n - 1$.\par
Например:
\begin{enumerate}
  \setcounter{enumi}{-1}
  \item Аппаратура.
  \item Планирование задач (планировщик).
  \item Управление памятью.
  \item Драйвер устройства связи оператора и консоли.
  \item Управление вводом/выводом.
  \item Интерфейс пользователя.
\end{enumerate}\par
Преимущества слоистых систем заключается в разделении функционала по слоям и
организации быстрой и удобной отладки и тестирования.

\subsection{Уравневые системы}
\begin{figure}[h!]
  \centering
  \begin{tikzpicture}[node distance=2cm]
    \node (pr_prog) [process] {Прикладная программа};
    \node (res_ser) [process, below of = pr_prog] {Резидентные системные сервисы};
    \node (dos) [process, below of = res_ser] {MS-DOS драйвера устройств};
    \node (BIOS) [process, below of = dos] {ROM-BIOS драйверы};
    \node (app) [process, below of = BIOS] {Аппаратура};
    \draw[arrow] (pr_prog) -- (res_ser);
    \draw[arrow] (res_ser) -- (dos);
    \draw[arrow] (dos) -- (BIOS);
    \draw[arrow] (BIOS) -- (app);
    \draw[arrow] (pr_prog.-7) -- +(0.5, 0) |- (BIOS);
    \draw[arrow] (pr_prog.east) -- +(1, 0) |- (app);
    \draw[arrow] (res_ser.+187) -- +(-0.5, 0) |- (BIOS);
    \draw[arrow] (res_ser.west) -- +(-1, 0) |- (app);
  \end{tikzpicture}
  \label{fig:}
\end{figure}

\subsection{Микроядерная (клиент-сервер)}
Перенос значительной части на уровень пользователя с одновременной минимизацией
ядра (создание микроядра). При такой микроядерной архитектуре операционной
системы, большинство ее модулей являются самостоятельными программами, а
взаимодействие между ними осуществляет специальный модуль ядра, называемый
микроядром.\par
Микроядро обеспечивает:
\begin{enumerate}
  \item Взаимодействие между программами.
  \item Планирование использования процессора (диспетчеризация).
  \item Первичную обработку прерываний.
  \item Операции ввода/вывода.
  \item Базовое управление памятью.
\end{enumerate}
Обработка прерывания:
\begin{enumerate}
  \item Идентификация прерывания.
  \item Устанавливается тип прерывания.
  \item Поиск обработчика по таблице и передача управления ему.
\end{enumerate}
