%! TEX root = ../main.tex
\section{Системное программное обеспечение ЭВМ}
Это комплекс программ, которые обеспечивает взаимодействие приложений пользователя с
аппаратурой и эффективное управление аппаратурой, к которой относятся:
\begin{itemize}
	\item Процессор\\{\ttfamily Регистры, контекст (из смены контекста), тактовая
		частота, элементарные операции (атомарные)}
	\item ОЗУ\\{\ttfamily Виртуальная память}
	\item Устройство ввода/вывода
	\item Сетевое оборудование
	\item Коммуникационное оборудование
\end{itemize}
В состав СПО входят 6 базовых компонентов:
\footnotetext[1]{СПО --- системное программное обеспечение.}
\begin{enumerate}
	\item ОС.
	\item Система управления файлами (СУФ).
	\item Интерфейсные оболочки для взаимодействия пользователя с ОС и
		операционные среды.
	\item Система программирования.
	\item Утилиты.
	\item СУБД (система управления базами данных).
\end{enumerate}\par
\textbf{ОС} --- базовый комплекс управляющих и обрабатывающих программ, которые управляют
аппаратно-программными ресурсами ЭВМ и задачами, при выполнении которых
используются эти ресурсы. ОС выполняет следующий задачи:
\begin{enumerate}
	\item Обеспечение работы пользовательских приложений и систем
		программирования.
	\item Прием и обработка пользовательских команд (в том числе с консоли).
	\item Прием и выполнение запросов на запуск, приостановку и остановку других
		программ.
	\item Загрузка программ подлежащих исполнению в оперативную память.
	\item Передача управление программе и выполнение программы процессором.
	\item Идентификация программ и данных.\\{\ttfamilyКаждому объекту должен
		сопоставляться собственный идентификатор.}
	\item Обеспечение работы системы управления файлами и системы управления
		базами данных.
	\item Управление операциями ввода/вывода.
	\item Распределение памяти.
	\item Диспетчеризация задач.\\{\ttfamilyВыборка задачи для смены контекста}
	\item Поддержка механизма обмена данными между исполняемыми программами.
	\item Защита памяти.
\end{enumerate}\par
\textbf{СУФ} --- система организации данных, хранения их и обращения к ним по
средствам файлов вместо низкоуровнего доступа по физическим адресам. Файл ---
цепочка кластеров во вторичной памяти. Кластер --- минимальлно адресуемая 
единица памяти 4\,кБ. Сектор~--- минимальная единица вторичной памяти
512\,Байт. С точки зрения ОС весь диск представляет из себя набор кластеров.\par
\textbf{Драйверы файловой системы} привязывают кластеры к файлам и
каталогам. \textbf{Каталог} --- файл специального формата, который содержит
список файлов в этом каталоге. Эти же драйверы отслеживают, какие из кластеров в
настоящие время используются, какие свободные, а какие помечены как неисправные.
Вместе с тем файловая система не обязательно напрямую связана с физическим
носителем информации. Существуют виртуальные и сетевые файловые системы, которые
являются всего лишь способом доступа к файлам, находящимся на удаленном
компьютере.\par
\textbf{Операционные среды} --- интерфейс необходимый прикладным программам
для обращения к системным ресурсам ОС с целью получения определенного сервиса.
Работа программной среды определяется прикладными программными интерфейсами ---
API. \textbf{API}~---~Application Program Interface. Примеры: Explorer, XWindow.
В семейство ОС Microsoft с интерфейсом Explorer заменяемой является только
интерфейсная оболочка, а операционная среда является неизменной. К этому классу
СПО относятся эмуляторы виртуальных машин (VMWare создает образ одной ОС на базе
другой).\par
\textbf{Система программирования} --- включает в себя:
\begin{enumerate}
	\item Трансляторы --- Специальные программы переводчики, которые переводят
		программы пользователей, написанные на различных ЯП, в машинный код. 3 вида:
		ассемблер, компиляторы (исходного модуля $\rightarrow$ объектный модуль), 
		интерпретаторы (системная программа, которая транслирует каждый оператор
		исходной прграммы в промежуточный код, интерпретирует его по средствам одной
		или нескольких команд и выполняет эти команды). 
	\item Библиотеки функций
	\item Редакторы
	\item Компоновщики
	\item Отладчики
	\item Специальные программы для выполнения вспомогательных функций.
\end{enumerate}
\subsection{Определение Ядра ОС}
Все модули ОС делятся на две группы
\begin{enumerate}
	\item Модули ядра. Включают в себя:
		\begin{enumerate}
			\item Планировщик (диспетчер)
			\item Драйверы устройств ввода/вывода
			\item Файловаю систему
			\item Сетевую систему
		\end{enumerate}
		Управляют задачами (потоками и процессами), памятью, устройствами и т.д.
		Функции такого типа являются внутрисистемными и недоступны для приложений.
		Ряд функций ядра служит для поддержки приложений, создавая для них, так
		называемую, прикладную программную среду.
	\item Модули выполняющие вспомогательные функции ОС (утилиты)
\end{enumerate}\par
Функции ядра являются наиболее часто используемыми функциями операционной
системы, поэтому скорость их выполнения определяет производительность всей
системы в целом. Для обепечения высокой скорости работы операционной системы все
модули ядра или большая их часть постоянно находятся в оперативной памяти
(являются резидентными) и не выгружаются оттуда. Такие модули операционной
системы, как утилиты, системные обработчики и библиотеки обычно загружаются в
оперативную память только на время выполнения своих функций (являются
транзитными). Ядо оформляется в видет отдельного модуля специального формата,
отличного от формата пользовательских приложений.\par
Ключевым свойством операционной системы, основанной на ядре, является защита
кодов и данных операционной системы засчет выполнения функций ядра в
привилегированном режиме.\par
Операционная система должна иметь по отношению к приложениям определенные
привилегии, то есть защиту от приложений, а также должна контролировать доступ
приложений к ресурсам компьютера в многозадачном режиме.\par
Ни одно приложение не должно иметь возможности без разрешения операционный 
системы получать дополнительную область памяти, занимать процессор дольше
разрешенного времени, а также непосредственно управлять совместно используемыми
внешними устройствами.\par
Привелегии обеспечиваются засчет специальных средств аппаратной поддержки. Для
этого аппаратная платформа должна поддерживать минимум два режима работы:
пользовательский и привилегированный. Соответственно приложение ставится в
подчиненное положение засчет запрета выполнения в пользоватльском режиме
некоторых критичных команд, например связанных с переключением процессора с
задачи на задачу, управлением устройствами ввода-ввывода, доступом к механизмам
распределения и защиты памяти.\par
\begin{figure}
	\centering
	\begin{tikzpicture} [node distance = 3cm]
		\node (name_var) [startstop] {Имена переменных};
		\node (vap) [startstop, below left of=name_var] {Виртуальные адреса};
		\node (phys) [startstop, below right of=vap] {Физические адреса};
		\draw [arrow] (name_var) -- (vap);
		\draw [arrow] (vap) -- (phys);
	\end{tikzpicture}
	\caption{Механизм распределения памяти.}
\end{figure}
Выполнение некоторых команд в пользовательском
режиме запрещается безусловно, тогда как другие команды запрещается выполнять
только при определенных условиях (пример: команды ввода-вывода могут быть
запрещены приложениям при доступе к контроллеру жесткого диска, который хранит
данные общие для операционной системы и всех приложений, но разрешены при
доступе к последовательному порту, который выделен в монопольное владение для
определенного приложения). Аналогичным образом обеспечиваются привилегии при
доступе к памяти (например: выполнение команды доступа к памяти для приложения
разрешается тогда, когда команда обращается к области памяти, которая отведена
данному приложению операционной системой, и запрещается при обращение к 
областям памяти занимаемым операционной системой или другим приложением).\par
Существуют две операции работы с памятью: выделение и резервирование. Полный
контроль операционной системы над доступом к памяти достигается засчет того, что
команды конфигурирования механизмов защиты памяти (наприер: изменение параметров
защиты) разрешается выполнять только в привелигированном режиме.

