%! TEX root = ../main.tex

\section{Лекция от 29.10.2022}
\subsection{Определение задач и ее свойства}
\texttt{Задачи лежат в структуре стека. Задача или процесс или поток. Ядро выбирает из
очереди задач ту, которую нужно загрузить первой в зависимости от свойств
задач.}\par

Задача --- набор операций для выполнения логических законченных функций
системы. Задачей может быть процессом или потоком.\par

Процесс --- независимый программный модуль, который во время выполнения имеет
отдельное адрессное пространство в памяти.\par

Поток --- операции ялвяющиеся составной частью процесса. Преимущества:
\begin{itemize}
  \item  Количество потоков не ограничено.
  \item Общее адресное пространство. 
  \item Время переключения между потоками гораздо меньше.
\end{itemize}
Минусы:
\begin{itemize}
  \item Общее адресное пространство. (Один поток никак не защищен от другого)
\end{itemize}
Свойства:
\begin{enumerate}
  \item Приоритет --- целое число. Может быть или статическим или динамическим.
    \texttt{Посмотреть, как приоритет потока зависит от приоритета процесса.}
\end{enumerate}\par

Дескриптор --- содержит ифнормацию о состоянии задачи, расположение образа задачи
на диске и в оперативной памяти, приоритете, идентификаторе пользователя
создавшего процесс, информацию о родственных процессах, о событиях осуществления
которых осуществляет данный процесс и так далее.\par

Состояния задачи:
\begin{itemize}
  \item Активное --- выполняется на процессоре.
  \item Блокированное --- находится в ожидании какого-либо события.
  \item Готовое --- ждет своей очереди.
\end{itemize}\par

Образ --- сочетание дескриптора и контекста. Контекст нужен, чтобы можно было
продолжить выполнение со снятого места. Дескрипторы объединяются в стек (списке).
\par

Контекст содержит менее оперативную, но более объемную информацию о задаче,
необходимую для возобновления выполнения задачи с прерванного места. Содержится:
\begin{enumerate}
  \item Содержимое регистров процессора.
  \item Коды ошибок выполняемых процессором системных вызовов.
  \item Информацию о файлах открытых в ходе выполнения данной задачи.
  \item Незавершенных операциях ввода/вывода.
  \item И другие данные описывающие состояние среды в момент прерывания.
\end{enumerate}

Контекст как и дескриптор доступен только программам ядра, но хранится не в
области ядра, а перемещается при необходимости из оперативной памяти на диск.
Блок TSB (именованная структура).
HANDLE --- id дескриптора.
