%! TEX root = ../main.tex

\section{Лекция от 13.09.2022}
\subsection{Человеческий мозг}
Человеческий мозг весит в среднем $1.3\,кг$.\par
\textbf{Нейрон} --- особый вид клеток, который обладает электрической
активностью. Он получает информацию при помощи сильно разветвленных отростков,
называемых \textbf{дендритами}. И передает информацию вдоль тонкого волокна,
называемого \textbf{аксоном}.\par
\textbf{Аксон} имеет множество ответвлений, на конце каждого из которых есть область,
называемая \textbf{синопсом}. С помощью синопсов осуществляется связь между
нейронами.\par
Каждый нейрон имеет тысячи связей с соседними нейронами. Информация по асонам
передается в виде коротких импульсов. На участках контактов мужду нейронами
(синопсисами) электрические импульсы превращаются в химические сигналы, которые
стимулируют проникновения в клетку положительных зарядов.\par
Когда достигается критческие значения потенциала, называемого \textbf{пороговым},
в ядре нейрона возникает электрический импульс, распространяемый как волна по
аксону на следующий нейрон. Вклад одного синопса в установление соответствующего
потенциала на выходе нейрона незначительно, поэтому для возникновения
электрического импульса необходимо, чтобы нейрон непрерывно интегрировал
множество синапсических входов. Такая интеграция является нелинейным
преобразованием.\par
Существуют:
\begin{itemize}
  \item Пирамидальные нейроны;
  \item Нейроны Таламуса;
  \item Нейроны Пуркинье;
  \item \dots
\end{itemize}
Всего около 50 видов нейронов. Совокупность нейронов и связей между ними
образуют \textbf{нейронную сеть}.
\subsection{Нейронная организация мозга}
Анатомически мозг разделен на ряд зон, выполняющих разные функции:
\begin{itemize}
  \item префронтальная кора;
  \item гипокамп;
  \item гипоталамус;
  \item зрительна кора;
  \item мембическая система.
\end{itemize}
Каждая область представляет собой гуппу нейронов различных типов, соединенных
между собой и другими частями мозга разнообразными связями.\par
\textbf{Лимбическая} система учавствует в эмоциональном поведении и 
долговременной памяти, которая хранит длительное время цифры, факты, правила и
события. Чтобы записи в этой памяти не забывались, необходимо через 
определенное время активизировать соответствующую нейронную структуру этой 
памяти. При повреждении нейронов долговременной памяти, человек утрачивает 
связь со своим прошлым.\par
\textbf{Гипокамп} выполняет функции кратковременной памяти, которая хранит
информацию без реактивации соответствующей нейронной сети от нескольких минут до
нескольких часов. Нейронная структура гипокампа перерабатывает и хранит новую
информацию, полученную в результате обучения и при соответствующей
реактивации.\par
\textbf{Префронтальная кора} участвует в образовании оперативной памяти, которая
необходима для извлечения фактов, событий и правил из долговременной памяти и
манипулирования ими или промежуточными результатами в соответствии с
обстоятельствами.\par
Из разделения мозга на различные зоны вытекает концепция функций организации
нейронных структур головного мозга. Согласно ей, различная информация
обрабатывается и хранится в разных нейронных сетях головного мозга. Так
некоторые исследователи считают, что префронтальная кора представляет собой
совокупность участков памяти, каждый из которых специализируется на информации
определенного рода. Например, цвет, размер объекта, семантические и
математические значения. При этом разные характеристики одного объекта
восприятия обрабатываются параллельно.
