%! TEX root = ../main.tex

\section{Лекция от 14.03.2023}

\textbf{Графы}
\begin{center}
  \begin{tikzpicture}[node distance = 2.5cm]
    \node[circle, draw=black] (top) {$c = c \cdot d$};
    \node[circle, draw=black, below left of = top] (sl) {$c = a + b$};
    \node[circle, draw=black, below right of = top] (sr) {$d = b + 1$};
    \node[circle, draw=black, below of = sl, minimum size = 1.5cm] (a) {$a$};
    \node[circle, draw=black, below of = sr, minimum size = 1.5cm] (b) {$b$};
    \draw[arrow] (a) -- (sl);
    \draw[arrow] (b) -- (sl);
    \draw[arrow] (b) -- (sr);
    \draw[arrow] (sl) -- (top);
    \draw[arrow] (sr) -- (top);
  \end{tikzpicture}
\end{center}
Пример
\[
  f(x_1, x_2) = 2^{x_1 x_2} - (x_1 + x_2)
\] 

\begin{center}
  \begin{tikzpicture}[node distance = 2.5cm]
    \node[circle, draw = black, minimum size = 1.5cm] (x1) {$x_1$};
    \node[circle, draw = black, left of = x1, minimum size = 1.5cm] (x2) {$x_2$};
    \node[circle, draw = black, minimum size = 1cm, above of = x1, xshift =
      0.5cm] (b) {$b = x_1 + x_2$};
    \node[circle, draw = black, above of = x2] (a) {$a = x_1 \cdot x_2$};
    \node[circle, draw = black, above of = a] (c) {$c = 2^a$};
    \node[circle, draw = black, above right of = c] (f) {$f = c -b$};
    \draw[arrow] (x1) -- (a);
    \draw[arrow] (x1) -- (b);
    \draw[arrow] (x2) -- (a);
    \draw[arrow] (x2) -- (b);
    \draw[arrow] (a) -- (c);
    \draw[arrow] (c) -- (f);
    \draw[arrow] (b) -- (f);
  \end{tikzpicture}
\end{center}

%\[
%  \begin{aligned}
%    &\frac{\partial f}{\partial c}  = \frac{\partial (c-b)}{\partial c} = 1\\
%    &\frac{\partial f}{\partial b}  = \frac{\partial (c-b)}{\partial b} = -1\\
%    &\frac{\partial f}{\partial a} = \frac{\partial f}{\partial c} \frac{\partial
%      c}{\partial a} = 1 \cdot 2^{a} \ln 2\\
%    &\frac{\partial f}{\partial x_1} = \frac{\partial f}{\partial a}
%      \frac{\partiall a}{\partial x_1} + \frac{\partial f}{\partial b}
%      \frac{\partial b}{\partial x_1} = x_2 \cdot 2^a \ln 2 + -1
%    & \frac{\partial f}{\partial x_2}  = \frac{\partial f}{\partial a}
%      \frac{\partial a}{\partial x_2}  + \frac{\partial f}{\partial b}
%      \frac{\partial b}{\partial x_2} = 
%  \end{aligned}
%\]

Алгоритм:
\begin{enumerate}
  \item Считаем частные производные по всем буквам входящим в корень
  \item Взять узел графа $V$  такой, что для всех его узлов сверху частные
    производные уже посчитаны.
  \item Тогда частные производные $\frac{\partial f}{\partial V}$ считаются по
    формуле
    \[
      \frac{\partial f}{\partial V} = \frac{\partial f}{\partial V_1}
      \frac{\partial V_1}{\partial V} + \ldots \frac{\partial f}{\partial V_k}
      \frac{\partial V_k}{\partial V} 
    \] 
    \[
      \frac{\partial f}{\partial V} = \sum_{i=1}^k \frac{\partial f}{\partial
      V_k} \frac{\partial V_k}{\partial V}  
    \] 
\end{enumerate}

Дан граф функции $f$:
\begin{center}
  \begin{tikzpicture}[node distance = 2.5cm]
    \node[circle, draw = black, minimum size = 1.5cm] (f) {$f = c^2$};
    \node[circle, draw = black, below of = f] (c) {$c = a+b$};
    \node[circle, draw = black, below left of = c, xshift = -1cm] (a) {$a = x_1 \cdot y_1$};
    \node[circle, draw = black, below right of = c, xshift = 1cm] (b) {$b = x_2 \cdot y_2$};
    \node[circle, draw = black, below left of = a, minimum size = 1.5cm] (x1) 
      {$x_1$};
    \node[circle, draw = black, below right of = a, minimum size = 1.5cm] (y1) 
      {$y_1$};
    \node[circle, draw = black, below left of = b, minimum size = 1.5cm] (x2) 
      {$x_2$};
    \node[circle, draw = black, below right of = b, minimum size = 1.5cm] (y2) 
      {$y_2$};
    \draw[arrow] (c) -- (f);
    \draw[arrow] (a) -- (c);
    \draw[arrow] (b) -- (c);
    \draw[arrow] (x1) -- (a);
    \draw[arrow] (y1) -- (a);
    \draw[arrow] (x2) -- (b);
    \draw[arrow] (y2) -- (b);
  \end{tikzpicture}
\end{center}

\[
  \begin{aligned}
    & \frac{\partial f}{\partial c} = 2c\\
    & \frac{\partial f}{\partial a} = \frac{\partial f}{\partial c}
      \frac{\partial c}{\partial a}  = 2c \cdot 1 = 2c \\
    & \frac{\partial f}{\partial b}  = \frac{\partial f}{\partial c} 
      \frac{\partial c}{\partial b} = 2c \cdot 1 = 2c \\
    & \frac{\partial f}{\partial x_1}  = \frac{\partial f}{\partial c} 
      \frac{\partial c}{\partial a} \frac{\partial a}{\partial x_1} 
      = 2c \cdot y_1 \\
    & \frac{\partial f}{\partial y1} = 2 c x_1
  \end{aligned}
\] 

\newpage
\paragraph{Задания:}\mbox{}\par
\begin{enumerate}
  \item есть лист графа $x$ далее идут $a := 2^{x}$ $b := 3^{x}$ $c := a + b$ 
    $d := ab$  $f := c/d$.\\
  Найти частные производные.
  \begin{center}
    \begin{tikzpicture}[node distance=2.5cm]
      \node[circle, draw=black, minimum size = 1.5cm] (x) {$x$};
      \node[circle, draw=black, above left of=x] (a) {$a = 2^{x}$};
      \node[circle, draw=black, above right of = x] (b) {$b = 3^{x}$};
      \node[circle, draw=black, above of = a] (c) {$c = a+b$};
      \node[circle, draw=black, above of = b] (d) {$d = a\cdot b$};
      \node[circle, draw=black, above left of = d] (f) {$f=\frac{c}{d}$};
      \draw[arrow] (x) -- (a);
      \draw[arrow] (x) -- (b);
      \draw[arrow] (a) -- (c);
      \draw[arrow] (b) -- (c);
      \draw[arrow] (a) -- (d);
      \draw[arrow] (b) -- (d);
      \draw[arrow] (c) -- (f);
      \draw[arrow] (d) -- (f);
    \end{tikzpicture}
  \end{center}
  \[
    \begin{aligned}
      & \frac{\partial f}{\partial c} = \frac{1}{d}\\
      & \frac{\partial f}{\partial d} = -\frac{1}{d^2} \\
      & \frac{\partial f}{\partial a} = \frac{\partial f}{\partial c}
        \frac{\partial c}{\partial a} + \frac{\partial f}{\partial d}
        \frac{\partial d}{\partial a} = \frac{1}{d} \cdot 1 - \frac{1}{d^2}
        \cdot b = \frac{d-1}{d^2} \\
      & \frac{\partial f}{\partial b} = \frac{\partial f}{\partial c} 
        \frac{\partial c}{\partial b} + \frac{\partial f}{\partial d} 
        \frac{\partial d}{\partial b} = \frac{1}{d}\cdot 1 - \frac{1}{d^2} a = 
        \frac{d - a}{d^2} \\
      & \frac{\partial f}{\partial x} = \frac{\partial f}{\partial a} 
        \frac{\partial a}{\partial x} + \frac{\partial f}{\partial b} 
        \frac{\partial b}{\partial x} = \frac{d-1}{d^2} \cdot 2^{x}\ln2 + 
        \frac{d-a}{d^2} 3^{x} \ln3 
    \end{aligned}
  \] 

  \item $x_1$  $x_2$  $a = x_1 \cdot x_2$  $b = x_1 + x_2$  $c = ab$  $d = a+b$ 
    $f = c^2 + d^2$

    \begin{center}
      \begin{tikzpicture}[node distance = 2cm]
        \node[ellipse, draw=black] (f) {$f = c^2 + d^2$}; 
        \node[ellipse, draw=black, below left of = f] (d) {$d = a+b$};
        \node[ellipse, draw=black, below right of = f] (c) {$c = a\cdot b$};
        \node[ellipse, draw=black, below of = d, xshift=-0.5cm] (a) 
          {$a=x_1 \cdot x_2$};
        \node[ellipse, draw=black, below of = c, xshift=0.5cm] (b) 
          {$b = x_1 + x_2$};
        \node[ellipse, draw=black, below of = a, minimum size = 1cm] (x1) {$x_1$};
        \node[ellipse, draw=black, below of = b, minimum size = 1cm] (x2) {$x_2$};
        \draw[arrow] (x1) -- (a);
        \draw[arrow] (x1) -- (b);
        \draw[arrow] (x2) -- (a);
        \draw[arrow] (x2) -- (b);
        \draw[arrow] (a) -- (c);
        \draw[arrow] (a) -- (d);
        \draw[arrow] (b) -- (c);
        \draw[arrow] (b) -- (d);
        \draw[arrow] (d) -- (f);
        \draw[arrow] (c) -- (f);
      \end{tikzpicture} 
    \end{center}
    \[
      \begin{aligned}
      & \frac{\partial f}{\partial d} = 2d \\
      & \frac{\partial f}{\partial c} = 2c \\
      & \frac{\partial f}{\partial a} = \frac{\partial f}{\partial d} 
        \frac{\partial d}{\partial a} + \frac{\partial f}{\partial c} 
        \frac{\partial c}{\partial a} = 2d \cdot 1 + 2c \cdot b \\
      & \frac{\partial f}{\partial b} = \frac{\partial f}{\partial d} 
      \frac{\partial d}{\partial b} + \frac{\partial f}{\partial c} 
      \frac{\partial c}{\partial b}  = 2d \cdot 1 + 2c \cdot a \\
      & \frac{\partial f}{\partial x_1} = \frac{\partial f}{\partial a} 
        \frac{\partial a}{\partial x_1} + \frac{\partial f}{\partial b}
        \frac{\partial b}{\partial x_1} = (2d + 2cb) \cdot (x_2) +
        (2d + 2ca) \cdot (1) = 28 \\
      & \frac{\partial f}{\partial x_2} = \frac{\partial f}{\partial a} 
      \frac{\partial a}{\partial x_2} + \frac{\partial f}{\partial b} 
      \frac{\partial b}{\partial x_2} = (2d + 2c b)\cdot (x_1) + 
      (2d + 2ca) = 24
      \end{aligned}
    \] 
\end{enumerate}


\newpage
\paragraph{Сигмоида}\mbox{}\par
Раньше сигмойда хайповала. Руки-базуки того времени.
\[
  \sigma(x) = \frac{e^{x}}{1 + e^{x}}
\] 
\[
  \sigma'(x) = \sigma(x) (1-\sigma(x)) 
\] 
Производная суперпозиции сигмойды
\[
  (\sigma(\sigma(x)))' = \sigma(\sigma(x)) (1-\sigma(\sigma(x)))\cdot
  \sigma(x)(1 - \sigma(x))
\] 
\[
  (\sigma(\sigma(\sigma(x))))' =
  \sigma(\sigma(\sigma(x)))(1-\sigma(\sigma(\sigma(x)))) \cdot
  (\sigma(\sigma(x)))'
\] 
При большом количестве производных сигмойда выраждается в ноль. (чем больше
слоев, тем больше производных)

Задание:\par
В точке $x = 0$. Найти 
\[
  \begin{aligned}
    & \sigma(0) = \frac{e^{0}}{1 + e^{0}} = 0.5 \\
    & \sigma'(0) = \sigma(0)(1-\sigma(0)) = 0.25 \\
    & \sigma'(\sigma(0)) = \sigma(\sigma(0)) (1 - \sigma(\sigma(0))) \cdot
      \sigma'(0) = \frac{e^{0.5}}{1+e^{0.5}} (1 - \frac{e^{0.5}}{1 + e^{0.5}}) 
      \cdot 0.25 = 0.235 \\
    & \sigma'(\sigma(\sigma(0))) = 0.227
  \end{aligned}
\] 
