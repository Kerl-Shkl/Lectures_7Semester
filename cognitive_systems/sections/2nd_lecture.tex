%! TEX root = ../main.tex

\section{Лекция от 14.02.2023}
Справка по первой работе:
\begin{enumerate}
  \item В cmdline ввести \verb!fuzzy!
  \item Выбрать функцию
  \item Задать границы
  \item Повторить для давления и выходной функции
  \item \verb!veiw! $\to$ \verb!rules!
  \item Поверхность отклика: \verb!view! $\to$ \verb!surface!
  \item \verb!edit! $\to$ \verb!rules!
\end{enumerate}

\subsection{Лингвистическая переменная}
Лингвистическая переменная характеризуется ее наименованием, множеством
значений, синтаксическими процедурами (позволяют оперировать элементами
терм-множества $T$), $U$  область определения.

Пример:
Переменная с именем температура в комнате

\begin{itemize}
  \item Синтаксические правила $G$, порождающее новые термы с использованием
    квантификаторов "и", "использованием, "новые"
\end{itemize}

Расмотри лингвистические переменные для первой работы:
{\ttfamily Принцип 80/20}
\begin{enumerate}
  \item Температура.
\end{enumerate}

Синглтон --- имеет форму прямой.
